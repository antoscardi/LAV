\chapter{Mathematical Model}

\section{Single Victim Case}
Three Cartesian coordinate frames are defined as \cite{main} and shown in \ref{fig:frames-1victim}:
\begin{enumerate}[label=(\roman*)]
    \item Frame $i$ (inertial): denoted as $F_i = (O_i, x_i, y_i, z_i)$, is the inertial frame with origin $O_i$.
    \item Frame $r$ (receiver ARTVA): denoted as $F_r = (O_r, x_r, y_r, z_r)$, is the body right-hand frame associated with the receiver installed on the drone.
    \item Frame $t$ (transmitter ARTVA): denoted as $F_t = (O_t, x_t, y_t, z_t)$, is the body right-hand frame associated with the transmitter worn by the victim.
\end{enumerate}
For the sake of simplicity, we assume that the body frame of the drone coincides with $F_r$. 
The position of $O_r$ relative to $O_t$ is indicated by the vector $\mathbf{p}_{tr} \in \mathbb{R}^3$, with $\mathbf{p}_{tr} = \mathbf{p}_r - \mathbf{p}_t$, 
while the positions of $O_r$ and $O_t$ relative to $O_i$ are indicated, respectively, by the vectors $\mathbf{p}_r \in \mathbb{R}^3$ and $\mathbf{p}_t \in \mathbb{R}^3$.
We use the apex $i$, $r$ or $t$ on the left of the vector to indicate in which frame the vector is expressed , e.g. ${}^t p$. 
If it is not specified, we assume the inertial frame.
\begin{figure}
    \centering
    \caption{Inertial frames in the single victim case}
    \label{fig:frames-1victim}
    \tdplotsetmaincoords{70}{110}
    \begin{tikzpicture}[tdplot_main_coords]
    \draw[thick,->] (0,0,0) -- (1,0,0) node[anchor=north east]{$y_i$};
    \draw[thick,->] (0,0,0) -- (0,1,0) node[anchor=north west]{$z_i$};
    \draw[thick,->] (0,0,0) -- (0,0,1) node[anchor=south]{$x_i$};
    % Point Pt1
    \coordinate (Pt1) at (5,0,5);
    \draw[->, thick] (0,0,0) -- (Pt1) node[pos=0.5,anchor=east]{${}^i \mathbf{p}_t$};      
    % Point tp
    \coordinate (p1) at (5,5,6);
    \draw[->, thick] (5,0,5) -- (p1) node[pos=0.5,anchor=south]{${}^t \bm{p}_{tr}$};      
    % pr
    \draw[->, thick] (0,0,0) -- (p1) node[pos=0.5,anchor=east]{${}^i \mathbf{p}_r$};

    \node at (p1) [anchor= south west] {$\mathbf{p}$};
    
    \coordinate (Shift) at (5,0,5);
    \tdplotsetrotatedcoordsorigin{(Shift)}
    \draw[thick,tdplot_rotated_coords,->] (0,0,0) --
    (.7,0,0) node[anchor=north]{$y_{t}$};
    \draw[thick,tdplot_rotated_coords,->] (0,0,0) --
    (0,.7,0) node[anchor=north]{$z_{t}$};
    \draw[thick,tdplot_rotated_coords,->] (0,0,0) --
    (0,0,.7) node[anchor=south]{$x_{t}$};
    \coordinate (Shift) at (5,5,6);
    \tdplotsetrotatedcoordsorigin{(Shift)}
    \draw[thick,tdplot_rotated_coords,->] (0,0,0) --
    (.7,0,0) node[anchor=east]{$y_{r}$};
    \draw[thick,tdplot_rotated_coords,->] (0,0,0) --
    (0,.7,0) node[anchor=west]{$z_{r}$};
    \draw[thick,tdplot_rotated_coords,->] (0,0,0) --
    (0,0,.7) node[anchor=south]{$x_{r}$};
    \end{tikzpicture}
\end{figure}
\subsection{Magnitude of Magnetic Field Intensity H}
We have found an expression of $\mathbf{H}$ in spherical coordinates, \ref{eq:H_spheric}, whose magnitude is found as:
\begin{equation}
    \left| \mathbf{H} \right| = \frac{I b^2}{4 \, r^3} \sqrt{ 4 \cos^2 \theta + \sin^2 \theta} = \frac{I b^2}{4 \, r^3} \sqrt{ 3 \cos^2 \theta + 1}
    \label{eq:H_mag_spher}
\end{equation}

\subsubsection{Approximation}
We use the same approximation in \cite{main} in order to remove the non-linearity given by the square root term $\sqrt{ 3 \cos^2 \theta + 1}$. Therefore we approximate:
\[
\frac{1}{\sqrt[3]{ 3 \cos^2 \theta + 1}} \approx \frac{1}{a^2 }\cos^2 \theta + \frac{1}{b^2} \sin^2 \theta
\]
of which the polar plot is shown in Figure \ref{fig:polarplot} when $a$ and $b$ have values 1.291 and 1.028, 
respectively, which minimize the relative mean squared error = 0.123\%.

% with the relative error this are instead the best values
% Best a: 1.284
% Best b: 1.034

Thus, the square root term becomes:
\begin{equation}
    \sqrt{ 3 \cos^2 \theta + 1} \approx \frac{1}{\left(\frac{1}{a^2} \cos^2 \theta + \frac{1}{b^2} \sin^2 \theta\right)^{3/2}}
    \label{eq:approx}
\end{equation}

Using the approximation \eqref{eq:approx} in \eqref{eq:H_mag_spher}:
\begin{equation}
    \left| \mathbf{H} \right| = \frac{I b^2}{4 \, r^3} \left(\frac{1}{a^2} \cos^2 \theta + \frac{1}{b^2} \sin^2 \theta\right)^{2/3}
    \label{eq:H_mag_approx}
\end{equation}

\begin{figure}[h!]
\centering
\includegraphics[width=0.5\textwidth]{images/polar_plot.png}
\caption{Polar plot of the actual function $\sqrt{ 3 \cos^2 \theta + 1} $ in blue 
and the approximated one $\frac{1}{a^2 }\cos^2 \theta + \frac{1}{b^2} \sin^2 \theta$ in orange.}
\label{fig:polarplot}
\end{figure}

Now we can express the magnitude using the Cartesian coordinates relative to the frame of the transmitter ARTVA $F_t$. 
If we consider the point ${}^t \mathbf{p}_{tr}$\[
    {}^t \mathbf{p}_{tr} = \begin{pmatrix}
    x \\
    y \\
    z
\end{pmatrix}
\] having these coordinates $(x,y,z)$ in frame $t$, then remembering $r$ from \ref{Conversion from Cartesian to Spherical Coordinates} and $\cos\theta$ from \ref{Conversion from Spherical to Cartesian Coordinates}:
\[
\begin{cases}
r^2 = x^2 + y^2 + z^2 \\
\cos\theta = \frac{x}{r}
\end{cases}
\]

Substituting the expressions for \(r\) and \(\cos\theta\) in \ref{eq:H_mag_approx}:
\[
\left| \mathbf{H} \right| = \frac{I b^2}{4} \frac{1}{(x^2 + y^2 + z^2)^{3}} \left(\frac{1}{a^2} \frac{x^2}{x^2 + y^2 + z^2} + \frac{1}{b^2} \frac{y^2 + z^2}{x^2 + y^2 + z^2}\right)^{2/3}
\]

After simplifications and further calculations, we obtain:
\begin{equation}
    \left| \mathbf{H} \right| = \frac{m}{4 \pi} \left(\frac{(ab)^2}{b^2 x^2 + a^2 (y^2 + z^2)}\right)^{3/2}
    \label{eq:H_mag_cart}
\end{equation}
where we call $I \, \pi \, b^2$ the magnetic moment $m$, as seen in \ref{eq:magnetic_moment}.
Then, we can define $\eta$ as \cite{main}:
\[ \eta = \left( \frac{m}{4 \pi \left| \mathbf{H} \right|} \right)^{2/3} \cdot \, (ab)^2 = \]
by substituting \ref{eq:H_mag_cart}:
\[
\begin{aligned}
&= \left( \frac{m}{4 \pi \frac{m}{4 \pi} \left(\frac{(ab)^2}{b^2 x^2 + a^2 (y^2 + z^2)}\right)^{3/2}} \right)^{2/3} \cdot (ab)^2 = 
\\
&= \left( \left(\frac{b^2 x^2 + a^2 (y^2 + z^2)}{(ab)^2}\right)^{3/2}\right)^{2/3} \cdot (ab)^2
\end{aligned}
\]

So:
\begin{equation}
    \eta = b^2 x^2 + a^2 (y^2 + z^2)
    \label{eq:eta}
\end{equation}

\subsection{Finding the ARTVA position}
In order to find the victim's position $\mathbf{p}_t$ with respect to the inertial frame $F_i$, we need to use homogeneous transformations \cite{book-robotics}. 
Also from Figure \ref{fig:frames-1victim}, we can express the position of the receiver $\mathbf{p}_r$ in the inertial frame as the sum of the other two vectors:
\[
\begin{aligned}
{}^i \mathbf{p}_r &= {}^i \mathbf{p}_t + {}^t \mathbf{p}_{tr} \\
{}^t \mathbf{p}_{tr} &= \mathbf{R}_t^i \,\, {}^i \mathbf{p}_{tr}
\end{aligned}
\]
where $\mathbf{R}_t^i$ is the rotation matrix that rotates axis $i$ to $t$ \cite{artva-gazebo}.
\begin{comment}
The matrix $R_t^i$ is the matrix which lets us express the vector in frame $i$ using the coordinates of the vector in frame $t$.Since we are expressing in the columns the coordinates of the axis of $t$ w.r.t using the axis of $i$.
Then if we have the coordinates of point $P$ with respect to frame i $\mathbf{P}$ 
(and if we know the rotation that goes from frame $i$ to frame $t$ (which also means we know the orientation and coordinates of the axis of frame $t$ with respect to those of frame $i$ $(R_t^i$ notation of Soper/this Chapman book), 
we can find the coordinates at point $P$ w.r.t frame i which is $\mathbf{P}$.
Also consider that from frame $i$ after a rotation $R_t^i$ we obtain the same orientation of frame $t$ which is just translated by the $\mathbf{P}_R$ vector.
v1 is  coordinates of the point before the translation but after the rotation, the coordinate of the point wrt the inertial frame  i are v1^i = R p^{i2}
\end{comment}

From which we can find $\mathbf{p}_t$ by multiplying by ${\mathbf{R}_t^i}^T$ since ${\mathbf{R}_t^i}$ is orthogonal (${\mathbf{R}_t^i}^T = {\mathbf{R}_t^i}^{-1}$):

\begin{equation}
    {}^t \mathbf{p}_{tr} = {\mathbf{R}_t^i}^T (\mathbf{p}_r - \mathbf{p}_t)
    \label{eq:pt}
\end{equation}

In addition, remembering how we defined the coordinates of ${}^t \mathbf{p}_{tr}$, then from linear algebra:
\[
\begin{aligned}
x &= \mathbf{e}_x^T \, {}^t \mathbf{p}_{tr} \\
y &= \mathbf{e}_y^T \, {}^t \mathbf{p}_{tr} \\
x  &= \mathbf{e}_z^T  \, {}^t \mathbf{p}_{tr}
\end{aligned}
\]
also,
\[
x^2 = x \cdot x = \left( \mathbf{e}_x^T \, {}^t \mathbf{p}_{tr} \right)^T \cdot \, \left( \mathbf{e}_x^T \, {}^t \mathbf{p}_{tr} \right)
\]
and the same is valid for the other two coordinates, we will omit the calculations for the other two from now on. 
We can then substitute the expression we found for ${}^t \mathbf{p}_{tr}$ \ref{eq:pt} and apply linear algebra properties of the transpose:
$$(\mathbf{A}\mathbf{B}\mathbf{C})^T = \mathbf{C}^T \mathbf{B}^T \mathbf{A}^T$$ to calculate the transpose of $\mathbf{e}_x^T {\mathbf{R}_t^i}^T (\mathbf{p}_r - \mathbf{p}_t)$:

\[
(\mathbf{e}_x^T {\mathbf{R}_t^i}^T (\mathbf{p}_r - \mathbf{p}_t))^T = (\mathbf{p}_r - \mathbf{p}_t)^T {\mathbf{R}_t^i} \mathbf{e}_x
\]
The expression for $x^2$ then becomes:
\begin{equation}
    x^2 = \left(\mathbf{p}_r - \mathbf{p}_t\right)^T {\mathbf{R}_t^i} \mathbf{e}_x \mathbf{e}_x^T {\mathbf{R}_t^i}^T \left(\mathbf{p}_r - \mathbf{p}_t\right)
    \label{eq:x2}
\end{equation}

Furthermore:
\[
\mathbf{e}_x \mathbf{e}_x^T =\begin{pmatrix}
1 \\
0 \\
0
\end{pmatrix}
\begin{pmatrix}
1 & 0 & 0
\end{pmatrix}
 = \text{diag}(1,0,0)
\]
and:
\[
\begin{aligned}
    \mathbf{e}_y \mathbf{e}_y^T &= \text{diag}(0,1,0) \\
    \mathbf{e}_z \mathbf{e}_z^T &= \text{diag}(0,0,1)
\end{aligned}
\]
Lastly we substitute the result found in \ref{eq:x2} in the expression of $\eta$ \ref{eq:eta}:
\[
\begin{aligned}
    \eta &=  \ b^2 \left(\mathbf{p}_r - \mathbf{p}_t\right)^T \mathbf{R}_t^i \mathbf{e}_x \mathbf{e}_x^T {\mathbf{R}_t^i}^T \left(\mathbf{p}_r - \mathbf{p}_t\right) + \\
    &+ a^2 \left(\mathbf{p}_r - \mathbf{p}_t\right)^T \mathbf{R}_t^i \mathbf{e}_y \mathbf{e}_y^T {\mathbf{R}_t^i}^T \left(\mathbf{p}_r - \mathbf{p}_t\right) +\\
    &+ a^2 \left(\mathbf{p}_r - \mathbf{p}_t\right)^T \mathbf{R}_t^i \mathbf{e}_z \mathbf{e}_z^T {\mathbf{R}_t^i}^T \left(\mathbf{p}_r - \mathbf{p}_t\right) =
\end{aligned}
\]
collect common terms,
\[
= \left(\mathbf{p}_r - \mathbf{p}_t\right)^T \mathbf{R}_t^i \left( b^2 \mathbf{e}_x \mathbf{e}_x^T + a^2 \mathbf{e}_y \mathbf{e}_y^T + a^2 \mathbf{e}_z \mathbf{e}_z^T \right) {R_t^i}^T \left(\mathbf{p}_r - \mathbf{p}_t\right)=
\]
\begin{equation}
    = \left(\mathbf{p}_r - \mathbf{p}_t\right)^T \mathbf{R}_t^i \, \text{diag}(b^2, a^2, a^2) {\mathbf{R}_t^i}^T \left(\mathbf{p}_r - \mathbf{p}_t\right)
    \label{eq:eta2}
\end{equation}

We call the \( \mathbf{R}_t^i \, \text{diag}(b^2, a^2, a^2) {\mathbf{R}_t^i}^T \) matrix $\mathbf{M}$ and the diagonal matrix $\text{diag}(b^2, a^2, a^2)$ $D$.

\subsubsection{Symmetry of \( \mathbf{M} \)}
\textbf{Definition} \\
A matrix \( \mathbf{M} \in \mathbb{R}^{n \times n} \) is symmetric if and only if \( \mathbf{M} = \mathbf{M}^T \).
\\
\textbf{Proof} \\
We compute \( \mathbf{M}^T \):
\[
    \mathbf{M}^T = \left( \mathbf{R}_t^i \mathbf{D} {\mathbf{R}_t^i}^T \right)^T = \left( {\mathbf{R}_t^i}^T \right)^T \mathbf{D}^T {\mathbf{R}_t^i}^T = \mathbf{R}_t^i \mathbf{D}^T {\mathbf{R}_t^i}^T
\]
Since \( \mathbf{D} \) is a diagonal matrix, it is equal to its transpose $\mathbf{D}^T = \mathbf{D}$ :
$$
\mathbf{M}^T =  \mathbf{R}_t^i D {\mathbf{R}_t^i}^T = \mathbf{M}
$$

\subsubsection{Final expression for $\eta$}
By applying the distributive property to \ref{eq:eta2} and since M is symmetric:
\begin{equation}
    \eta = \mathbf{p}_r^T \mathbf{M} \mathbf{p}_r - \mathbf{p}_r^T \mathbf{M} \mathbf{p}_t - \mathbf{p}_t^T \mathbf{M} \mathbf{p}_r+ \mathbf{p}_t^T \mathbf{M} \mathbf{p}_t
    \label{eq:eta3}
\end{equation}
The vector $\mathbf{\hat{p}}_t$ gives an estimate of the true position $\mathbf{p}_t$ :
\[
\mathbf{\hat{p}}_t = \mathbf{M} \mathbf{p}_t
\]
and since $M$ is symmetric:
\[
\mathbf{\hat{p}}_t^T =  \mathbf{p}_t^T \mathbf{M}^T = \mathbf{p}_t^T \mathbf{M}
\]
We substitute these expressions in \ref{eq:eta3} and use the definition of the scalar product:
\[
\eta = \mathbf{p}_r^T \mathbf{M} \mathbf{p}_r - \mathbf{p}_r^T \mathbf{\hat{p}}_t - \mathbf{\hat{p}}_t^T \mathbf{p}_r+ \mathbf{p}_t^T \mathbf{M} \mathbf{p}_r = \mathbf{p}_r^T \mathbf{M} \mathbf{p}_r - 2 \mathbf{p}_r^T \mathbf{\hat{p}}_t + \mathbf{p}_t^T \mathbf{M} \mathbf{p}_t
\]
If we define the coordinates of $\mathbf{p}_r$ :
\[
\mathbf{p}_r = \begin{pmatrix}
    x_r \\
    y_r \\
    z_r
\end{pmatrix}
\]
and
\[
    \mathbf{M} = \begin{pmatrix}
    m_{11} & m_{12} & m_{13} \\
    m_{12} & m_{22} & m_{23} \\
    m_{13} & m_{23} & m_{33} \\
\end{pmatrix}
\]
we can compute $\mathbf{p}_r^T \mathbf{M} \mathbf{p}_r$:
\[
\mathbf{p}_r^T \mathbf{M} \mathbf{p}_r = 
m_{11} \, x_r^2 \; + \; 2 \, m_{12} \, x_r \, y_r \; + \; 2 \, m_{13} \, z_r \, x_r \; + \;
m_{22} \, y_r^2 \; + \; 2 \, m_{23} \, y_r \, z_r \; + \; m_{33} \, z_r^2
\]
Then, we obtain a final expression for $\eta$ as in \cite{main}:
\begin{equation}
\begin{aligned}
\eta = & \ m_{11} \, x_r^2 + 2 \, m_{12} \, x_r \, y_r + 2 \, m_{13} \, z_r \, x_r \\
       & \ + m_{22} \, y_r^2 + 2 \, m_{23} \, y_r \, z_r + m_{33} \, z_r^2 \\
       & \ - 2 \, x_r \, x_t - 2 \, y_r \, y_t - 2 \, z_r \, z_t \\
       & \ + \mathbf{p}_t^T \mathbf{M} \mathbf{p}_t
\end{aligned}
\label{eq:eta_final}
\end{equation}

\subsection{Finding the NSS}
In the only one victim case, we can find the the NSS not only numerically but also
analytically.
Which means that since we have an expression of the magnetic field intensity $\mathbf{H}$
in Cartesian coordinates \ref{eq:final_H}, we can compute the derivatives and therefore the
gradient tensor $\mathbf{G}$ analytically as:
\begin{equation}
\mathbf{G} = \frac{1}{r^{5/2}}
\scalebox{0.96}{$
\begin{bmatrix}
3 x (-2 x^2 + 3y^2 + 3z^2) & 3 y (-4 x^2 + y^2 + z^2) & 3 z (-4x^2 + y^2 + z^2) \\
3 y (-4 x^2 + y^2 + z^2) & 3 x (x^2 - 4 y^2 + z^2) & -15 x y z \\
3 z (-4 x^2 + y^2 + z^2) & -15 x y z & 3 x (x^2 + y^2 - 4 z^2)
\end{bmatrix}
$}
\label{eq:G_analit}
\end{equation}
where $r$ is of course the distance, $r = \sqrt{x^2 +y^2 +z^2}$.

Note once again the symmetry and traceless property of the gradient tensor 
$\mathbf{G}$ is evident, when we compute it explicitly.
Also note that for simplicity we computed the gradient using $\mathbf{H}$ and not $\mathbf{B}$,
since they vary only by the constant permeability $\mu$. 
We also omitted the constants $I$, $b$, since they don't depend on the coordinates $(x, y, z)$
neither, and therefore do not influence the results.

In Figure \ref{fig:gradients_single_anal}, we show the gradients for a single
source centered at the origin of the space, when the inertial frame 
$F_i$ coincides with the trasmitter (source) frame $F_t$ and also when there 
is no rotation between these frames and the receiver frame $F_r$ of the drones. 
In this way, we have that the magnetic field intensity ${}^t \mathbf{H} = {}^i \mathbf{H}$
can be computed using the inertial coordinates, and also the
NSS at any point in space coincides with the signal received by 
any drone located at that point.
The gradients are calculated on a grid of equally spaced points in the range $[-5, 5]$, when we fix the $z$ coordinates
at 3 m, considering a reasonable flying height for the drones.
We need to fix one coordinate in order to be able to plot 
a function of 3 variables, but also because we assume the drones
flying always at a fixed height, as it will be explained more in 
depth lately.   
\begin{figure}
\centering
\includegraphics[width=\textwidth]{images/gradients_single_anal.jpg}
\caption{Plot of the gradients $\frac{\partial H_i}{\partial i}$ where $i = x, y, z$
which are the componets of the gradient tensor $\mathbf{G}$
when computed analytically as in \ref{eq:G_analit}, in the case of a single source located
at the center $(0,0)$ of the space when there are no rotations between the coordinates
frames $F_i$, $F_r$, $F_t$.}
\label{fig:gradients_single_anal}
\end{figure}

After having found the gradient tensor $\mathbf{G}$, we numerically
compute its eigenvalues and we find the NSS using the definition \ref{eq:NSS}.
In Figure \ref{fig:NSS_single_anal}, we plot the NSS distribution over the same grid,
with the same $z$ value fixed.
These values will be the signals received by the drones at their location
in space in the multiple victims case, which will be explained more in depth in the following paragraph.
\begin{figure}
\centering
\includegraphics[width=0.8\textwidth]{images/NSS_single_anal.jpg}
\caption{Plot of the NSS values computed at each point on the grid, which
represent the signal received by the drones, in the case of a single source located
at the center $(0,0)$ of the space when there are no rotations between the coordinates
frames $F_i$, $F_r$, $F_t$..}
\label{fig:NSS_single_anal}
\end{figure}

Instead, let's consider the case when the source location is translated
with respect to the inertial frame $F_i$ and the transmitter frame $F_t$ is rotated
with respect to the inertial by a rotation $\mathbf{R}_t^i$, like in the previous 
section.
Furthermore, we also consider the rotation $\mathbf{R}^t_r$ between the transmitter frame and the receiver
frame $F_r$.
Then, we compute the coordinates of any point in space with respect to the
transmitter frame as in \ref{eq:pt}.
Using these coordinates we find the magnetic intensity vector ${}^t\mathbf{H}$,
expressed in the trasmitter frame and compute the gradient tensor ${}^t \mathbf{G}$.
Then, as we defined the $\mathbf{R}^i_t$ the matrix that rotates axis $i$ to $t$,
we find the gradient tensor ${}^i \mathbf{G}$ by inverting equation \ref{eq:rotated_gradient_tensor}:
\begin{equation}
    {}^i \mathbf{G} = {\mathbf{R}_t^i}^T \, {}^t\mathbf{G} \, \mathbf{R}_t^i
\end{equation}
In Figure \ref{fig:gradients_rotated_single_anal}, we see how the gradients plots
vary with respect to the gradients in Figure \ref{fig:gradients_single_anal},
since the rotations introduce naturally a modification in the elettromagnetic field.
However, in Figure \ref{fig:NSS_rotated_single_anal} it is also evident how the
NSS distribution remains invariant under all the different rotations, and it is 
just affected by the translation in the sense that the peak now is located
at the new location of the source. 
% MATRICI USATE PER LA ROTAZIONE
%R_array{1} = rotationMatrix(70, 0, 0);
%R_drones{1} = rotationMatrix(170, 0, 0);
\begin{figure}
\centering
\includegraphics[width=\textwidth]{images/gradients_rotated_single_anal.jpg}
\caption{Plot of the gradients $\frac{\partial H_i}{\partial i}$ where $i = x, y, z$
when computed analytically after the position of the signle source is translated to $(2,2)$ and an 
introduction of the different roations, expressed using $\mathbf{R}^t_r$ and $\mathbf{R}_t^i$.}
\label{fig:gradients_rotated_single_anal}
\end{figure}
\begin{figure}
    \centering
    \includegraphics[width=0.8\textwidth]{images/NSS_rotated_single_anal.jpg}
    \caption{Plot of the NSS after the position of the signle source is translated to $(2,2)$ and 
    different roations are introduced, expressed using $\mathbf{R}^t_r$ and $\mathbf{R}_t^i$.}
    \label{fig:NSS_rotated_single_anal}
\end{figure}

\noindent
\\
However, for the multiple victims case we cannot compute the gradient tensor 
analytically and therefore we adopt a numerical approach.
We simulate and simplify the way the authors of \cite{NSS_single_localization} use 
a cubic structure measurement array composed of eight
tri-axial magnetometers in order to obtain the magnetic
field information and therefore compute the gradient tensor $\mathbf{G}$.
Let's consider a small perturbation $\delta$ in order to apply the central difference
method approximation.
We perturb the magnetic field intensity $\mathbf{H}$ by every direction 

The gradient of a 3D vector field \( \mathbf{f}(\mathbf{p}) \), where \( \mathbf{p} = (x, y, z) \), 
can be numerically approximated using the central difference method. Let \( \delta \) be a small 
perturbation applied to each coordinate direction. The partial derivative of \( \mathbf{f} \) with 
respect to a coordinate \( p_k \) is given by:

\[
\frac{\partial \mathbf{f}}{\partial p_k} \approx \frac{\mathbf{f}(\mathbf{p} + \delta \mathbf{e}_k) 
- \mathbf{f}(\mathbf{p} - \delta \mathbf{e}_k)}{2\delta},
\]

where \( \mathbf{e}_k \) denotes the unit vector in the \( k \)-th direction. The gradient matrix, 
or Jacobian \( \mathbf{J} \), is constructed by repeating this process for each coordinate direction:

This method, based on the Taylor series expansion, yields a second-order accurate estimate of 
the gradient by approximating the derivative as a difference of function values at perturbed points.

For a small perturbation \( \delta \), let \( \boldsymbol{\delta}_k \) be a vector with \( \delta \) in the \( k \)-th position. The perturbed points are:
\[
\mathbf{p}^{+}_k = \mathbf{p} + \boldsymbol{\delta}_k, \quad \mathbf{p}^{-}_k = \mathbf{p} - \boldsymbol{\delta}_k.
\]

The \( k \)-th column of \( \mathbf{J} \) is:
\[
\mathbf{J}_{:, k} = \frac{\mathbf{f}(\mathbf{p}^{+}_k) - \mathbf{f}(\mathbf{p}^{-}_k)}{2 \delta}.
\]

General Formula:
\[
\frac{\partial f_i}{\partial p_k} \approx \frac{f_i(\mathbf{p}^{+}_k) - f_i(\mathbf{p}^{-}_k)}{2 \delta}.
\]


\section{Multiple Victim Case}
%https://math.stackexchange.com/questions/4761965/addition-of-vectors-in-different-coordinate-systems
%https://www2.physics.ox.ac.uk/sites/default/files/2011-10-08/coordinates_pdf_51202.pdf
\begin{figure}
    \centering
    \caption{Only 2 victims case}
    \label{fig:frames-2victim}
    \tdplotsetmaincoords{70}{110}
    \begin{tikzpicture}[tdplot_main_coords]
    \draw[thick,->] (0,0,0) -- (1,0,0) node[anchor=north east]{$y_i$};
    \draw[thick,->] (0,0,0) -- (0,1,0) node[anchor=north west]{$z_i$};
    \draw[thick,->] (0,0,0) -- (0,0,1) node[anchor=south]{$x_i$};
    % Point Pt1
    \coordinate (Pt1) at (5,0,5);
    \draw[->, thick] (0,0,0) -- (Pt1) node[pos=0.5,anchor= west]{$\mathbf{p}_{t_1}$};
    % pr
    \draw[->, thick] (0,0,0) -- (p1) node[pos=0.5,anchor=west]{$\mathbf{p}_r$};
    % Point Pt2
    \coordinate (Pt2) at (0,5,3);
    \draw[->, thick] (0,0,0) -- (Pt2) node[pos=0.5,anchor= south]{$\mathbf{p}_{t_2}$};
    % Point t1p
    \coordinate (p1) at (5,5,6);
    \draw[->, thick] (5,0,5) -- (p1) node[pos=0.5, anchor=south]{$\stackrel{t_1}{\phantom{p}\bm{p}}$};      
    \node at (p1) [anchor=south] {$\bm{p}$};    
    % Point t2p
    \coordinate (p2) at (5,5,6);
    \draw[->, thick] (0,5,3) -- (p2) node[pos=0.5, anchor=south]{$\stackrel{t_2}{\phantom{p}\bm{p}}$};
    \node at (p2) [anchor=south] {$\bm{p}$};
    % Vectors H1 and H2
    \coordinate (H1) at (3,1.5,3); 
    \coordinate (H2) at (1,3,3); 
    \draw[->, thick, blue] (p1) -- (H1) node[anchor=north west]{$\mathbf{H}_1$};
    \draw[->, thick, red] (p1) -- (H2) node[anchor=north]{$\mathbf{H}_2$};
    % Angle arc
        \pic ["$\alpha$",draw = black, ->,
        angle radius=7mm,
        angle eccentricity=1.3, 
        ] { angle = H1--p1--H2};
    \coordinate (Shift) at (5,0,5);
    \tdplotsetrotatedcoordsorigin{(Shift)}
    \draw[thick,tdplot_rotated_coords,->] (0,0,0) --
    (.7,0,0) node[anchor=north]{$y_{t_1}$};
    \draw[thick,tdplot_rotated_coords,->] (0,0,0) --
    (0,.7,0) node[anchor=west]{$z_{t_1}$};
    \draw[thick,tdplot_rotated_coords,->] (0,0,0) --
    (0,0,.7) node[anchor=south]{$x_{t_1}$};
    \coordinate (Shift) at (0,5,3);
    \tdplotsetrotatedcoordsorigin{(Shift)}
    \draw[thick,tdplot_rotated_coords,->] (0,0,0) --
    (.7,0,0) node[anchor=north]{$y_{t_2}$};
    \draw[thick,tdplot_rotated_coords,->] (0,0,0) --
    (0,.7,0) node[anchor=west]{$z_{t_2}$};
    \draw[thick,tdplot_rotated_coords,->] (0,0,0) --
    (0,0,.7) node[anchor=south]{$x_{t_2}$};
    \end{tikzpicture}
\end{figure}

In the multiple victim case an ARTVA transmitter is attached to every one of the \( n \) avalanche victims, 
therefore each receiver is affected by \( n \) generated electromagnetic fields.
In Figure \ref{fig:frames-2victim}, the two victims case is represented, using the same frames as the single victim one. 
For brevity we will call \( R_1 \) and \( R_2 \) the relative orientations of frames \( t_1 \) and \( t_2 \) with respect to the reference frame \(i\).

Now, if the receiver is positioned at point \( \mathbf{p} \) in space, 
the summed effect of the electromagnetic fields can be expressed as the sum of the magnetic field intensity vector fields \(\mathbf{H}_n\):
\[
\mathbf{H}_{\text{tot}} = 
\mathbf{H}_1 + \mathbf{H}_2 + \cdots + \mathbf{H}_n
\]
For the single magnetic field intensity vector of the \(i\)-th victim, remembering \ref{eq:H_spheric}:
\[
\mathbf{H}_i = \frac{I b^2}{4\pi r^3} \left( \mathbf{e}_{r_i} \, 2 \cos \theta + \mathbf{e}_{\theta_i} \, \sin \theta \right)
\]
Furthermore, we use the same homogeneous transformations as \ref{eq:pt}:
\[
\mathbf{p}_{t_1} + R_1 \, \mathbf{p}^{t_1} = \mathbf{p}_r
\]
\[
\mathbf{p}_{t_2} + R_2 \, \mathbf{p}^{t_2} = \mathbf{p}_r
\]
which become again:
\[
\mathbf{p}^{t_1} = R_1^T \left( \mathbf{p}_R - \mathbf{p}_{t_1} \right)
\]
\[
\mathbf{p}^{t_2} = R_2^T \left( \mathbf{p}_R - \mathbf{p}_{t_2} \right)
\] 

\subsection{Total Magnitude }
Suppose the case of only two victims, the results can be then generalized.
If \( \alpha \) is the angle between the two vectors \( \mathbf{H}_1 \) and \( \mathbf{H}_2 \) on the only plane which contains both vector fields, we have:
\[
||\mathbf{H}_{\text{tot}}||^2 = ||\mathbf{H}_1||^2 + ||\mathbf{H}_2||^2 + 2 \, ||\mathbf{H}_1|| \, ||\mathbf{H}_2|| \cos \alpha
\]
Substituting the formula of the magnetic field magnitude after the approximation, with the Cartesian coordinates of reference frame $F_t$ \ref{eq:H_mag_cart}:
\[
\begin{aligned}
||\mathbf{H}_{\text{tot}}||^2 &= \left( \frac{m}{4 \pi} \right)^2 \left( \frac{(ab)^2}{b^2 \, x_1^2 + a^2 \, (y_1^2 + z_1^2)} \right)^3 
+ \left( \frac{m}{4 \pi} \right)^2 \left( \frac{(ab)^2}{b^2 \, x_2^2 + a^2 \, (y_2^2 + z_2^2)} \right)^3 + \\
& + 2 \, \cos \alpha \left( \frac{m}{4 \pi} \right)^2 \left( \frac{(ab)^2}{b^2 \, x_1^2 + a^2 \, (y_1^2 + z_1^2)} \right)^{3/2} \left( \frac{(ab)^2}{b^2 \, x_2^2 + a^2 \, (y_2^2 + z_2^2)} \right)^{3/2} = \\
&= \left( \frac{m \, (ab)^3}{4 \pi} \right)^2 \left( \left( \frac{1}{b^2 \, x_1^2 + a^2 \, (y_1^2 + z_1^2)} \right)^3 
+  \left( \frac{1}{b^2 \, x_2^2 + a^2 \, (y_2^2 + z_2^2)} \right)^3 \right. + \\
& \left. + \, 2 \, \cos \alpha \left( \frac{1}{(b^2 \, x_1^2 + a^2 \, (y_1^2 + z_1^2))(b^2 \, x_2^2 + a^2 \, (y_2^2 + z_2^2))} \right)^{3/2} \right)
\end{aligned}
\]
Note that a and b, have the same values for all the fields since they have been chosen in order to optimize the magnitude on any magnetic field intensity. 
Also we suppose that the radius $b$ and the current $I$ are the same for all the devices (so same magnetic moment $m$).

Bringing the common terms to the left side and inverting all the fractions:
\[
\begin{aligned}
\left( \frac{m \, (a b)^3}{4 \pi \, ||\mathbf{H}_{\text{tot}}||} \right)^2 &= \left( b^2 \, x_1^2 + a^2 \, (y_1^2 + z_1^2) \right)^3 + \left( b^2 \, x_2^2 + a^2 \, (y_2^2 + z_2^2) \right)^3 + \\
& + 2 \, \cos \alpha \left( (b^2 \, x_1^2 + a^2 \, (y_1^2 + z_1^2))(b^2 \, x_2^2 + a^2 \, (y_2^2 + z_2^2)) \right)^{3/2}
\end{aligned}
\]


SPIEGARE PERchE QUESTO È UNFEASIBLE



\subsection{Finding the NSS}
