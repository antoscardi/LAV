\chapter{}

\cite{DetectionMultiplMagnetiDipoleSources}
This article proposes a new method to simultaneously estimate the locations and magnetic moments of multiple magnetic dipole sources without the prior knowledge of the number of dipoles in the 3-D detection region. By initializing a large number of dipole sources evenly spaced in the detection region as potential candidates for the true dipoles, we introduce an indicator parameter for each dipole candidate such that its Sigmoid function is the probability that the candidate converges to a true dipole. A joint optimization is then formulated to minimize the mean square of the regularized error between the measured magnetic gradients and the calculated gradients from the estimated dipoles. The proposed nonlinear optimization is solved by the Levenberg–Marquardt algorithm, yielding the indicators and their corresponding dipole locations and magnetic moments.
\\

\cite{multiple_in_spacecraft}
Multi-magnetic source resolution in spacecraft has been a difficult problem in the field of magnetic surveys. Detection technology of magnetic gradient tensor is available to solve this issue due to its high resolution and precision. A spacecraft magnetic source model is established, and a multi-magnetic source model fitting method for spacecraft is presented. The principal invariants of the magnetic field gradient tensor are introduced to determine the number and horizontal location of the sources, while Euler equations are used to compute the source depth, achieving a resolution of up to 0.012m, which meets engineering requirements.
\\

\cite{real_application}
Nothing much is a real application.
\\

\cite{multiple_plots}
Questo è il paper utile dove ci sono tutti i plot.
\\

\cite{multiple_DE_NSS_big_matrix}
This article presents a new method for detecting and localizing multiple dipole-like magnetic sources using magnetic gradient tensor data. The tilt angle (ratio of vertical to horizontal magnetic field components) is used to determine the number of sources, while the rotational-invariant normalized source strength (NSS) is used to estimate the horizontal coordinates. The Differential Evolution (DE) algorithm estimates the locations and moments of the sources.
\\

\cite{single_closed_formula_position}
This paper presents a simple formula for the localization of a magnetic dipole. First, the position vector is derived from the analytical expressions of the magnetic field vector and the magnetic gradient tensor. The proposed algorithm provides the true position of the dipole regardless of the singularity of the magnetic gradient tensor matrix.
\\

\cite{multiple_real_plots_invariants}
This paper proposes a new edge detection method using magnetic gradient tensor components for magnetic exploration, which is free from geomagnetic interference and provides abundant information. The method is compared with others, such as THDz, AS, tilt angle, and theta map, under various conditions. The experimental results show that the proposed method is more precise and delivers high-quality edge detection with strong anti-interference capabilities.
\\

\cite{NSS_formula_important}
This paper proposes a two-point magnetic gradient tensor localization model to overcome errors caused by geomagnetic fields. The model uses the spatial relation between the magnetic target and observation points derived from tensor invariants. A new method is presented for accurately locating magnetic targets, achieving nearly error-free results in the absence of noise.
\\

\cite{NSS_single_different_dimensions}
This paper introduces new methods for inverting magnetic gradient tensor data to obtain source parameters for various models, such as dipoles and thin sheets. Eigenvalues and eigenvectors of the tensor are used in combination with normalized source strength (NSS) to uniquely determine source locations. NSS analysis is extended to vertical pipes by calculating eigenvalues of the vertical derivative of the tensor.
\\

\cite{NSS_single_localization}
This article introduces a novel magnetic dipole localization method based on normalized source strength (NSS) to overcome asphericity errors in magnetic anomaly detection (MAD). A closed-form localization formula is derived, and an optimization method is proposed to improve noise immunity. Simulation and field experiments demonstrate high localization accuracy, real-time performance, and robustness against noise and misalignment errors.
\\

\cite{NSS_analysis}
For a number of widely used models, normalized source strength (NSS) can be derived from eigenvalues of the magnetic gradient tensor. NSS is proportional to a constant normalized by the distance between observation and integration points. It is independent of magnetization direction and satisfies Euler’s homogeneity equation, allowing for Euler deconvolution of the NSS to estimate source location. The method was applied to aeromagnetic data from the Tuckers Igneous Complex, Queensland, Australia, improving the interpretation of magnetic anomalies with strong remanent magnetization.
 \\

 \cite{not_import_formula_z}
 closed loop formula to find z 
 \\

 \cite{NSS_analysis2}
 Recent technological advances suggest that we are on thethreshold of a new era in applied magnetic surveys,where acquisition of magnetic gradient tensor data willbecome routine. In the meantime, modern ultrahighresolution conventional magnetic data can be used, withcertain important caveats, to calculate gradient tensorelements from total magnetic intensity (TMI) or TMIgradient surveys. Until the present, not a great deal ofattention has been paid to processing and interpretationof gradient tensor data. New methods for invertinggradient tensor surveys to obtain source parameters havebeen developed for a number of elementary, but useful,models. These include point pole, line of poles, pointdipole (sphere), line of dipoles (horizontal cylinder), thinand thick dipping sheets, sloping step and contactmodels. A key simplification is the use of eigenvaluesand associated eigenvectors of the tensor. The scaledsource strength, calculated from the eigenvalues, is aparticularly useful rotational invariant that peaks directlyover compact sources, 2D sources and contacts,independent of magnetisation direction. New algorithmsfor uniquely determining the location and magneticmoment of a dipole source from a few irregularly locatedmeasurements or single profiles have been developed.Besides the geological applications, these algorithms arereadily applicable to the detection, location andclassification (DLC) of magnetic objects, such as navalmines, UXO, shipwrecks, archaeological artefacts andburied drums. As an example, some of these newmethods are applied to analysis of the magnetic signatureof the Mount Leyshon gold-mineralised system,Queensland.
 \\

 \cite{PSO_IMPORTANT}
 In this paper, a modified particle swarm optimization (PSO) algorithm is developed for solving multimodal function optimization problems. The difference between the proposed method and the general PSO is to split up the original single population into several subpopulations according to the order of particles. The best particle within each subpopulation is recorded and then applied into the velocity updating formula to replace the original global best particle in the whole population. To update all particles in each subpopulation, the modified velocity formula is utilized. Based on the idea of multiple subpopulations, for the multimodal function optimization the several optima including the global and local solutions may probably be found by these best particles separately. To show the efficiency of the proposed method, two kinds of function optimizations are provided, including a single modal function optimization and a complex multimodal function optimization. Simulation results will demonstrate the convergence behavior of particles by the number of iterations, and the global and local system solutions are solved by these best particles of subpopulations.


 