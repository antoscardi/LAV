\documentclass[main]{subfiles}
\begin{document}
\section*{Ringraziamenti}
\addcontentsline{toc}{section}{Ringraziamenti}
Desidero ringraziare vivamente il mio relatore, il professore \textbf{Andrea Cristofaro}, per avermi dato la possibilità di
approfondire questo interessante argomento, per avermi seguito e supportato nella realizzazione di questo progetto e
per i consigli da lei ricevuti. La ringrazio profondamente.

\noindent\\
Ringrazio inoltre la professoressa \textbf{Marilena Vendittelli} per il tempo dedicato alla revisione della relazione e per 
i suggerimenti che hanno contribuito a migliorare il risultato finale.

\noindent\\
A \textbf{Mamma} e \textbf{Papà}, per quasi tutta la mia vita ho dovuto scegliere o è stato scelto per me la precedenza dell'uno o dell'altro, 
quindi questa volta scelgo io.
E scelgo di ringraziarvi insieme, del supporto che mi avete dato e dei sacrifici che avete fatto
per farmi raggiungere questo traguardo. Non ci sono parole per descrivere il bene che vi voglio.

\noindent\\
A \textbf{nonno Minuccio}, non so se esista un luogo in cui ti trovi e da cui mi guardi. So solo che sei nel mio cuore,
insieme al ricordo più bello della mia infanzia e forse anche della mia vita. 

\noindent\\
A \textbf{nonno Camillo}, mi hai insegnato una delle qualità più importanti nella vita, la pazienza.
Ancora non riesco a credere che tu non ci sia più. 

\noindent\\
A \textbf{nonna Rosetta}, sei stata e sei ancora come una mamma per me. Ti voglio bene.

\noindent\\
A \textbf{nonna Marianna}, per il coraggio e la saggezza che mi hai trasmesso. Ti voglio bene.

\noindent\\
A \textbf{zia Marisa} e \textbf{zio Mimmo}, per esserci sempre stati e per avermi sempre considerato come un figlio. Vi voglio bene.

\noindent\\
A \textbf{zio Giuseppe} e \textbf{zia Maria Rita}, siete per me il senso del Natale. Vi voglio bene.

\noindent\\
A \textbf{zio Salvatore} e \textbf{zia Luisa}, per avermi sempre accolto e cercato. Vi voglio bene.

\noindent\\
A \textbf{Lella}, \textbf{Nichi} e \textbf{Ari}, una sorella maggiore, una minore e un fratello che non ho. Vi voglio bene.

\noindent\\
A \textbf{Cami}, per il bene fraterno che mi hai sempre dimostrato. Ti voglio bene.

\noindent\\
A \textbf{Prisci}, seconda di sangue ma prima di cuore. Ti voglio bene.

\noindent\\
A \textbf{Carmine} e \textbf{Checco}, grazie per tutto quello che avete fatto per me. Vi voglio bene.

\noindent\\
A \textbf{Stefano}, non so come tu faccia a sopportarmi. Grazie per accettarmi interamente, 
con tutti i miei difetti. Ti amo.

\noindent\\
A \textbf{Gian}, mio fratello bianco, sei la mia sicurezza. Ti voglio bene.
\noindent\\
A \textbf{Federico}, grazie per la tua simpatia e per la tua parte migliore, l'empatia. Ti voglio bene.
\noindent\\
A \textbf{Ferri}, grazie per ascoltarmi e capirmi sempre. Ti voglio bene.
\noindent\\
A \textbf{Matteo}, è inutile che ripeto, sai già tutto. Ti voglio bene.
\noindent\\
A \textbf{Valerio}, il mio pandino, hai un mondo nel cuore. Ti voglio bene.
\noindent\\
A \textbf{Sandrino}, grazie per esserci stato nel momento più difficile di questo percorso. Ti voglio bene.
\noindent\\
A \textbf{Asta}, grazie per avermi mostrato che ci si può abbandonare. Ti voglio bene.
\noindent\\
A \textbf{Gigio}, per i tanti anni passati assieme, comunque ti voglio bene.
\noindent\\
A \textbf{Proio}, \textbf{Francesco}, e \textbf{Aurora}, per esserci stati durante il periodo più
brutto durante il Covid.
\noindent\\
A \textbf{L' oro} e a \textbf{L' odio}, e tutti i loro membri, di ognuno mi porto qualcosa.
Spero un giorno si possano riunificare. 

\noindent\\
A \textbf{Manfri}, per come sei e per il bene che mi vuoi. Ti voglio bene.
\noindent\\
A \textbf{Mire}, il mio alterego. Ti voglio bene.
\noindent\\
A \textbf{Lollo} e \textbf{Michi}, grazie Lollo per avermi fatto conoscere Michi. A parte gli scherzi, grazie Lollo
per il tuo sostegno. Vi voglio bene. 
\noindent\\ 
A \textbf{Carletto}, il mio compagno di banco e mio fratello. Ti voglio bene.
\noindent\\
A \textbf{Sofi}, la migliore amica, perchè ci sei sempre e mi hai sempre sostenuto con il tuo affetto. Ti voglio bene.

\noindent\\
A \textbf{Nicola}, \textbf{Marco} e \textbf{Ludovico}, vi metto assieme per dispetto, e per ricordarvi dell'adolescenza
che abbiamo passato insieme. Grazie \textbf{Nico} per avermi fatto provare la prima canna. Sei un coglione, nel senso più buono.
Grazie \textbf{Marco} per avermi fatto scoprire me stesso e grazie \textbf{Ludo} per esserci sempre stato. Vi voglio bene.

\noindent\\
A \textbf{Dante}, anche se lontani e poco frequenti, quando ci incontriamo sembra non sia passato un giorno. Ti voglio bene.
\noindent\\
A \textbf{Giovanni}, e ai momenti passati alle Mole, liberi e felici. 

\noindent\\
A \textbf{Martina}, non è facile, troppe cose hai fatto per me nelle ultime due settimane, non sto qui a elencarle.
Mi sento di essere scesi nel profondo, ti voglio bene è poco.
\noindent\\
A \textbf{Mattia}, il colpo di fulmine, alla fine ci siamo capiti, mio bellissimo bimbo biondo. Ti voglio bene.
\noindent\\
A \textbf{Gabriele}, la prima luce nel momento più buio della mia vita. Ti voglio bene.
\noindent\\
A \textbf{Nick}, abbiamo fatto troppi progetti insieme, ma li rifarei tutti. Ti voglio bene.
\noindent\\
A \textbf{Massimo}, mi fai troppo ridere. Ti voglio bene.
\noindent\\
A \textbf{Peppe}, e alle due teste e Senseri. Ti voglio bene.
\noindent\\
A \textbf{Gerardo}, il Grande Gigante Gentile. Ti voglio bene.
\noindent\\
A \textbf{Marco}, grazie per le mozzarelle e per come sei. Ti voglio bene.
\noindent\\
A \textbf{Diego}, mi rivedo in te, crediamo più in noi stessi. Ti voglio bene.
\noindent\\
A \textbf{Pasquale}, sei buono e bimbo dentro (scambiami gli ex). Ti voglio bene.
\noindent\\ 
A \textbf{Daniel} e \textbf{Jacopo}, \textbf{Richi} e \textbf{Antonio} e \textbf{Luca},
viva la sburra. Vi voglio bene.
\noindent\\
A \textbf{Rossana} e \textbf{Francesca}, grazia e graziella, manca grazie al cazzo. Vi voglio bene.
\noindent\\
A \textbf{Vincenzo}, \textbf{Anna} e \textbf{Catia}, per aver patito insieme a me
il momento decisivo, vi voglio bene.

\noindent\\
A \textbf{Castelluccio}, e ai Lucani, dal greco \textit{lykos} ``lupo'',  
o dal latino \textit{lucus} ``bosco sacro'', oppure dalla radice indoeuropea \textit{leuk}  
(ereditata sia dal greco ``bianco'' che dal latino ``luce'').  
Siete tanti e non potrei nominarvi tutti senza dimenticare qualcuno.  
Grazie per accogliermi sempre, per me siete Casa.  
La terra non può voler male all’albero.
\noindent\\
A \textbf{Gaia}, per me rappresenti Milano, se un pezzo della mia vita. Ti voglio bene.
\noindent\\
A \textbf{Raff}, continua a prenderti cura di Nicola. Ti voglio bene.

\noindent\\
A \textbf{Giulia}, per non essersi dimenticata di me, nonostante la distanza. Ti voglio bene.
\noindent\\
A \textbf{Edo}, per aver sempre creduto in me, compagno di merende. Ti voglio bene.
\noindent\\ 
A \textbf{Max} e \textbf{Andre C.}, \textbf{Andre M.} e \textbf{Pippo}, per aver fatto parte 
del gruppo più bello del Politecnico. Mi mancate.

\noindent\\
A \textbf{Ruggero}, \textbf{Diana}, \textbf{Asterio}, \textbf{Carlo} e \textbf{Davide},
mi avte fatto conoscere me stesso, mi mancate.

\noindent\\
A \textbf{Giovanna}, grazie per avermi fatto capire chi sono Io.
A \textbf{Vincenzo} e \textbf{al Gruppo}, senza di voi non sarei mai riuscito a raggiungere questo traguardo.
Non trovo altre parole, vi voglio bene.

\noindent\\
A \textbf{Renato}, detto \textbf{René}, detto \textbf{Mimmi}, non puoi leggere perché sei un gatto. Solo un pazzo ringrazierebbe un gatto.
Beh, ormai si sa che sono pazzo, ma tu rappresenti il mio essere più profondo e quindi mi ringrazio per aver resistito 
fino alla fine. 

\end{document}
