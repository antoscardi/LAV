\documentclass[main]{subfiles}
\begin{document}
\section{Theoretical Foundation}

\subsection{Gradient, Divergence, and Curl} \label{Gradient, Divergence, and Curl}
We briefly define the following operators which will be used throughout this 
work.

\subsubsection{Gradient of a Scalar Field}
\textbf{Definition}\noindent\\
Given a scalar function \( f(x_1, x_2, \ldots, x_n): \mathbb{R}^n \to \mathbb{R} 
\) or scalar field, the gradient of the function is a vector field of partial 
derivatives and denoted by \( \nabla f \), is defined as:
\begin{equation}
\nabla f = \left( \frac{\partial f}{\partial x_1}, \frac{\partial f}{\partial 
x_2}, \frac{\partial f}{\partial x_3}, \ldots, \frac{\partial f}{\partial x_n} 
\right)
\label{eq:gradient}
\end{equation}
where \( \nabla \) is the vector differential operator \cite{math-book}.

\noindent\\
\textbf{Properties}\noindent\\
The gradient is a vector which points in the direction of the greatest ascent of 
the function and its magnitude is the increase.

\subsubsection{Definition of a Vector Field}
A \textbf{vector field} on a subset \(S \subseteq \mathbb{R}^n\) is a 
vector-valued function \(\mathbf{V}: S \to \mathbb{R}^n\) that assigns to each 
point \(\mathbf{x} = (x_1, x_2, \ldots, x_n) \in S\) a vector 
\(\mathbf{V}(\mathbf{x})\) \cite{math-book}.

\noindent\\
Note that from now on, we usually omit the spatial dependency of vector fields
from points $\mathbf{x}$, since it's implied.

\subsubsection{Divergence of a Vector Field}
For simplicity and since we are working with \(\mathbb{R}^3\) Euclidean space, we 
limit our discussion from now on to Euclidean coordinates.

\noindent\\
\textbf{Definition}\noindent\\
Given a vector field \(\mathbf{V}\), the divergence of \(\mathbf{V}\) at a point 
\( \mathbf{p} \in \mathbb{R}^3 \) is defined as the net outward flux of 
\(\mathbf{V}\) per unit volume \(\Delta v\) as the volume about the point tends 
to zero:
\begin{equation}
\nabla \cdot \mathbf{V} = \lim_{\Delta v \to 0} \frac{1}{\Delta v} \iint_S 
\mathbf{V} \cdot d\mathbf{s}
\label{eq:divergence}
\end{equation}
where \( \mathbf{V} \cdot d\mathbf{s} \) is the flux of \( \mathbf{V} \) through 
the surface \(S\) \cite{book-magnetism}.

\noindent\\
Since \(\mathbf{V} = V_{x} \mathbf{e_x} + V_{y} \mathbf{e_y} + V_{z} 
\mathbf{e_z}\), the divergence of \(\mathbf{V}\), can be computed as:
\begin{equation}
\nabla \cdot \mathbf{V} = \frac{\partial V_{x}}{\partial x} + \frac{\partial 
V_{y}}{\partial y} + \frac{\partial V_{z}}{\partial z}
\end{equation}
where \(V_{x}\), \(V_{y}\), and \(V_{z}\) are the components of the vector field 
\(\mathbf{V}\) in the \(x\), \(y\), and \(z\) directions, respectively 
\cite{math-book}.

\noindent\\
\textbf{Properties}\noindent\\
When the vector field is represented using flux lines (indicating the direction 
and intensity), the divergence is the number of flux lines diverging/converging 
through a given point.

\noindent\\
A net outward flux of a vector field through a surface bounding a volume 
indicates the presence of a source, the divergence measures the strength of the 
source. The divergence of a vector field is a scalar field. 

\subsubsection{Laplacian of a Vector Field}
The Laplacian of a vector field $\mathbf{V}$, denoted by $\nabla^2$, is similar 
to the scalar Laplacian and is defined as:
\begin{equation}
\nabla^2 V = \frac{\partial^2 V_x}{\partial x^2} + \frac{\partial^2 V_y}{\partial 
y^2} + \frac{\partial^2 V_z}{\partial z^2} \quad
\end{equation}

\noindent\\
Alternatively, by taking the curl of the curl of a vector field, the Laplacian 
can be expressed as:
\begin{equation}
   \nabla^2 \mathbf{V} = \nabla (\nabla \cdot \mathbf{V}) - \nabla \times 
 (\nabla \times \mathbf{V})
   \label{eq:laplacian}
\end{equation}

\subsubsection{Curl of a Vector Field}
\textbf{Definition}\noindent\\
The curl of \(\mathbf{V}\), denoted by \(\nabla \times \mathbf{V}\), at a point in 
space \(\mathbf{x} \in \mathbb{R}^3\) is a vector field whose magnitude is the 
maximum net circulation of \(\mathbf{V}\) per unit area as the area tends to zero 
and whose direction is the normal direction of the area when the area is oriented 
to make the net circulation maximum:
\begin{equation}
\nabla \times \mathbf{V} = \lim_{\Delta s \to 0} \frac{1}{\Delta s} \mathbf{n} 
\oint_{C} \mathbf{V} \cdot d\mathbf{l}
\label{eq:curl}
\end{equation}
where \(\mathbf{n}\) is the unit normal vector to the surface \(S\), \(d\mathbf{l}\) 
is the differential line element along the boundary, and the integral represents 
the circulation of \(\mathbf{V}\) around the boundary of the surface 
\cite{book-magnetism}.

\noindent\\
The curl of a vector field \(\mathbf{V}\) can be then computed in terms of its 
components \cite{math-book}:
\begin{equation}
\nabla \times \mathbf{V} = \left( \frac{\partial V_z}{\partial y} - \frac{\partial 
V_y}{\partial z}, \frac{\partial V_x}{\partial z} - \frac{\partial V_z}{\partial 
x}, \frac{\partial V_y}{\partial x} - \frac{\partial V_x}{\partial y} \right)
\label{eq:curl_calculation}
\end{equation}

\noindent\\
\textbf{Properties}\noindent\\
Since the normal to an area can point in two opposite directions, the direction 
of the curl is given by the right-hand rule.
A vortex source causes a circulation of the vector field around it. The 
circulation of a vector field around a closed path is defined as the scalar line 
integral of the vector over the path. Note that a circulation of \(\mathbf{V}\) 
can exist even when the divergence of \(\mathbf{V}\) is zero, meaning there is no 
net source or sink.

\subsection{Stokes' Theorem} \label{Stokes' Theorem}
The surface integral of the curl of a vector field \(\mathbf{V}\) over a surface 
\( s \) is equal to the line integral of the vector field over the boundary 
contour \( c \) of the surface:
\begin{equation}
\oint_C \mathbf{V} \cdot d\mathbf{l} = \iint_S (\nabla \times \mathbf{V}) \cdot 
d\mathbf{s}
\label{eq:stokes}
\end{equation}
The proof of the theorem comes directly from the definition of the curl 
\eqref{eq:curl} and by dividing the surface $S$ into smaller areas. The idea comes 
from the fact that computing the line integral around the boundary of a surface 
is equal to computing the integral for all the smaller areas, since the 
\(d\mathbf{l}\) components of the neighboring regions are in opposite 
directions.

\subsection{Divergence Theorem}
The theorem states that the surface integral of a vector field \(\mathbf{V}\) 
over a closed surface \(S\) is equal to the volume integral of the divergence of 
\(\mathbf{V}\) over the volume enclosed by \(S\):
\begin{equation}
\iint_S\mathbf{V} \cdot d\mathbf{s} = \iiint_V (\nabla \cdot \mathbf{V}) \, dv
\label{eq:divergence_th}
\end{equation}

\noindent\\
The idea of the proof is similar to \ref{Stokes' Theorem}, starting from the 
definition of the divergence \eqref{eq:divergence}. Considering a volume divided 
into smaller volumes, the contributions from the internal surfaces cancel each 
other, leaving only the contribution from the outer surface.

\subsection{Null Identity Theorem}
The divergence of the curl of any vector field is always zero \cite{book-magnetism}:
\begin{equation}
\nabla \cdot (\nabla \times \mathbf{V}) = 0 
\label{eq:null}
\end{equation}
The proof leverages the Divergence Theorem \eqref{eq:divergence_th} applied to the 
vector field \(\nabla \cdot (\nabla \times \mathbf{V})\). Considering that any 
volume can be divided in half, then the surface bounding the volume would be the 
sum of 2 surfaces, connected by a common boundary that has been drawn twice. One 
can then compute the two surface integrals using Eq.\ref{eq:stokes}, and since the 
two normals \(\mathbf{n}\) have equal intensity and opposite direction, their sum 
is zero and the integrand as well.

\noindent\\
A converse statement of the theorem is as follows: 
If a vector field \(\mathbf{B}\) is divergence-less, then it can be expressed as 
the curl of another vector field \(\mathbf{V}\):
\[
\nabla \cdot \mathbf{B} = 0 \implies \mathbf{B} = \nabla \times \mathbf{V}
\]
Since the identity \(\nabla \cdot (\nabla \times \mathbf{V}) = 0\) always holds, 
it means that for any magnetic field \(\mathbf{B}\) with zero divergence, we can 
find a vector potential \(\mathbf{V}\) such that \(\mathbf{B}\) is the curl of 
\(\mathbf{V}\).

\subsection{Phasor Notation} \label{Phasor Notation}
For dealing with time-varying vector fields, we use phasor notation to represent 
sinusoidal varying field vectors (which is usually the case in real-world 
applications, such as the magnetic dipole) \cite{book-magnetism}. Phasor notation 
simplifies the analysis of such fields by converting differential equations into 
algebraic equations.

\noindent\\
Then, an harmonic vector field \( \mathbf{V}(x, y, z, t) = \mathbf{V}(x, y, z) 
\, \cos\omega t\) can be represented by a vector phasor that depends on space 
coordinates but not on time:
\[
\mathbf{V}(x, y, z, t) = \mathbf{V}(x, y, z) e^{j\omega t}
\]
where \(\omega\) is the angular frequency and \( \mathbf{V}(x, y, z) \) is a 
vector phasor that contains information on direction, magnitude, and phase.

\noindent\\
The time-domain function can be recovered from the phasor by taking the real part:
\[
\mathbf{V}(x, y, z, t) = \Re\{ \mathbf{V}(x, y, z) e^{j\omega t} \}
\]

\subsection{Spherical Coordinates} \label{Spherical Coordinates}
We define two classical coordinate systems: spherical coordinates $(r, \theta, 
\phi)$ and Cartesian coordinates \((x, y, z)\), with their respective vector 
basis \( \{\mathbf{e}_r, \mathbf{e}_{\theta}, \mathbf{e}_{\phi} \} \) and 
\( \{\mathbf{e}_x, \mathbf{e}_y, \mathbf{e}_z\} \), represented in Figure 
\ref{fig:coordinates}, from which the following relationships are evident. Note 
that we use the convention in \cite{main} for the angles and the axes. 

\begin{figure}
    \centering
    \tdplotsetmaincoords{70}{110}
    \pgfmathsetmacro{\thetavec}{48.17}
    \pgfmathsetmacro{\phivec}{63.5}
    
    \begin{tikzpicture}[tdplot_main_coords]
        %Axis
        \draw[axis] (0,0,0) -- (6.5,0,0) node [pos=1.1] {$y$};
        \draw[axis] (0,0,0) -- (0,6,0) node [pos=1.05] {$z$};
        \draw[axis] (0,0,0) -- (0,0,5.5)  node [pos=1.05] {$x$};   
        %Unit Vectors
        \tdplotsetcoord{P'}{7}{\thetavec}{\phivec}
        \draw[univec] (0,0,0) -- (P') node [pos=1.05] {$\mathbf{e_r}$};
        \tdplotsetcoord{P''}{1}{90}{90+\phivec}
        \draw[univec] (2,4,0) -- ($(P'') + (2,4,0)$) node [pos=1.3] {$\mathbf{e_\phi}$};
        \tdplotsetcoord{P'''}{1}{90+\thetavec}{\phivec}
        \draw[univec] (2,4,4) -- ($(P''') + (2,4,4)$) node [pos=1.3] {$\mathbf{e_\theta}$};
        
        %Vectors
        \tdplotsetcoord{P}{6}{\thetavec}{\phivec}
        \draw[vec] (0,0,0) -- (P) node [midway, above] {$r$};
        \draw[thick] (0,0,0) -- (2,4,0);
        
        %Help Lines
        \draw[dashed] (2,4,4) -- (2,4,0);
        \draw[dashed] (2,0,0) -- (2,4,0) node [pos=-0.1] {};
        \draw[dashed] (0,4,0) -- (2,4,0) node [pos=-0.3] {};
        \draw[dashed] (0,0,4) -- (2,4,4) node [pos=-0.1] {};
        \draw[dashed, tdplot_main_coords] (4.47,0,0) arc (0:90:4.47);
        
        %Point
        \node[fill=black, circle, inner sep=0.8pt] at (2,4,4) {};
        
        %Angles
        \tdplotdrawarc{(0,0,0)}{0.7}{0}{\phivec}{below}{$\phi$}
         
        \tdplotsetthetaplanecoords{\phivec}
        \tdplotdrawarc[tdplot_rotated_coords]{(0,0,0)}{0.5}{0}{\thetavec}{}{}
        \node at (0,0.25,0.67) {$\theta$};
        
    \end{tikzpicture}
    \caption{Spherical and Cartesian coordinates with their respective unit 
 vectors.}
    \label{fig:coordinates}
\end{figure}

\subsubsection{Conversion from Cartesian to Spherical Coordinates} 
\label{Conversion from Cartesian to Spherical Coordinates}
\begin{align}
 r &= \sqrt{x^2 + y^2 + z^2}, \\
    \theta &= \cos^{-1} \left(\frac{\sqrt{z^2 + y^2}}{x}\right), \\
    \phi &= \tan^{-1} \left(\frac{y}{z}\right).
\end{align}

\subsubsection{Conversion from Cartesian to Spherical Coordinates} 
\label{Conversion from Spherical to Cartesian Coordinates}
\begin{align}
 x &= r \cos \theta, \\
 y &= r \sin \theta \cos \phi, \\
 z &= r \sin \theta \sin \phi.   
\end{align}

\subsubsection{Expressing Spherical Unit Vectors using Cartesian Unit Vectors}
A point $\mathbf{r}$ in Cartesian coordinates is given by:
\[
\mathbf{r} = x \, \mathbf{e}_x + y \, \mathbf{e}_y + z \, \mathbf{e}_z.
\]
The radial unit vector \( \mathbf{e}_r \) is defined as the normalized position 
vector, obtained by substituting the coordinates in Section
\ref{Conversion from Cartesian to Spherical Coordinates}:
\begin{equation}
\mathbf{e}_r = \frac{\mathbf{r}}{|\mathbf{r}|} = \cos\theta \, \mathbf{e}_x + 
\sin\theta \cos\phi \, \mathbf{e}_y + \sin\theta \sin\phi  \, \mathbf{e}_z.
\label{e_r}
\end{equation}
The unit vector \( \mathbf{e}_\theta \), which is perpendicular to \( 
\mathbf{e}_r \) and lies in the plane formed by the origin and the \( z \)-axis, 
points in the direction of increasing \( \theta \). It can be found by taking the 
partial derivative of the position vector with respect to \( \theta \) and 
normalize it:
\begin{equation}
\mathbf{e}_\theta = -\sin\theta \, \mathbf{e}_x + \cos\theta \cos\phi \, 
\mathbf{e}_y + \cos\theta \sin\phi \, \mathbf{e}_z.
\label{e_theta}
\end{equation}
Similarly, the unit vector \( \mathbf{e}_\phi \), which is perpendicular to both 
\( \mathbf{e}_r \) and \( \mathbf{e}_\theta \), points in the direction of 
increasing \( \phi \). It can be derived by taking the partial derivative of the 
position vector with respect to \( \phi \) and normalizing it:
\[
\mathbf{e}_\phi = \frac{\frac{\partial \mathbf{r}}{\partial 
\phi}}{|\frac{\partial \mathbf{r}}{\partial \phi}|} = - \sin\phi  \, \mathbf{e}_y 
+ \cos\phi \, \mathbf{e}_z.
\]

\subsubsection{The Gradient, Divergence, and Curl in Spherical Coordinates}
We omit the demonstration of how these formulas are found starting from the 
definitions given in Cartesian coordinates \ref{Gradient, Divergence, and Curl}, 
which involve a large amount of computations \cite{book-magnetism}.

\noindent\\
For a vector field expressed in spherical coordinates:
\[
\mathbf{V} = V_r \mathbf{e}_R + V_\theta \mathbf{e}_{\theta} + V_\phi \mathbf{e}_\phi 
\]
The divergence is:
\begin{equation}
\nabla \cdot \mathbf{V} = \frac{1}{r^2} \frac{\partial}{\partial r} (r^2 V_r) + 
\frac{1}{r \sin \theta} \frac{\partial}{\partial \theta} (\sin \theta V_\theta) 
+ \frac{1}{r \sin \theta} \frac{\partial V_\phi}{\partial \phi}.
\end{equation}
The curl in spherical coordinates is:
\begin{equation}
\nabla \times \mathbf{V} = \frac{1}{r \sin \theta} \begin{vmatrix} 
 \mathbf{e}_r & r  \mathbf{e}_\theta & r \sin \theta \mathbf{e}_\phi \\ 
 \frac{\partial}{\partial r} & \frac{\partial}{\partial \theta} & 
 \frac{\partial}{\partial \phi} \\ 
 V_r & r V_\theta & r \sin \theta V_\phi 
    \end{vmatrix}.
    \label{curl_spher}
\end{equation}
carrying out the determinant:
\begin{equation}
\begin{aligned}
\nabla \times \mathbf{V} = \frac{1}{r \sin \theta} \Bigg[ 
 \left( \frac{\partial}{\partial \theta} (V_\phi \sin \theta) - 
 \frac{\partial V_\theta}{\partial \phi} \right)  &\mathbf{e}_r \\
 + \left( \frac{1}{\sin \theta} \frac{\partial V_r}{\partial \phi} - 
 \frac{\partial}{\partial r} (r V_\phi) \right)  &\mathbf{e}_\theta \\
 + \left( \frac{\partial}{\partial r} (r V_\theta) - \frac{\partial V_r}{\partial 
    \theta} \right)  &\mathbf{e}_\phi 
\Bigg].
\end{aligned}
\end{equation}

\subsection{3D Dirac Delta}
\textbf{Definition}\noindent\\
Given a point \(\mathbf{r}\) in Cartesian coordinates, the Dirac delta function 
\(\delta(\mathbf{r})\) is defined as \cite{book-magnetism2}:
\begin{itemize}
    \item \(\delta(\mathbf{r}) = 0\) at all points except at \(\mathbf{r} = 
 (0,0,0)\).
    \item The integral across the entire space satisfies:
    \begin{equation}
    \int_V \delta(\mathbf{r}) \, dv = 1
    \label{eq:dirac}
    \end{equation}
\end{itemize}

\noindent\\
The result of the integral could be the value of any function in zero, $f(0)$.
For example, the Dirac delta could be the charge density $\rho$ defined in 
Table \ref{tab:symbols} 
of a point particle located at the origin whose charge is \(q\).

\subsection{Green's Function for Poisson Equation}
Poisson's equation using Green's function is written as:
\begin{equation}
\nabla^2 G(\mathbf{r}) = \delta(\mathbf{r})
\label{eq:poisson}
\end{equation}
The solution to the above equation is given by:
\[
G(r) =  - \frac{1}{4 \pi r}
\]
\textbf{Proof}\noindent\\
We assume \(G(r)\) to be axis-symmetric, implying that it only depends on the 
magnitude $r$, not the vector position $\mathbf{r}$.
Considering the point \((x,y,z)\) and the distance \(r\) from the origin.
To find Green's function we look for the simplest function that satisfies the 
equation, which has the form:
\[
G(r) = A \frac{1}{r} + B
\]
where \(A\) and \(B\) are constants. 

\noindent\\
Assuming \(B = 0\) for simplicity, we find \(A\) by integrating over a sphere of 
volume $\epsilon$.
Substituting Poisson's Equation \eqref{eq:poisson} in the definition of the 
Dirac delta \eqref{eq:dirac}:
\[
\int_V \nabla^2 G(r) \, dv = 1 
\]
Using the Divergence Theorem \eqref{eq:divergence_th}:
\[
\int_V \nabla^2 G(r) \, dv = \int_S \nabla G(r) \cdot d\mathbf{s} 
\]
\[
1 = \int_S \nabla G(r) \cdot d\mathbf{s} 
\]
The right-hand side represents the flux through the surface \(S\). We take the 
integral over the surface of a sphere of radius \(r\), knowing that the surface 
area is \(4 \pi r^2\) and computing the divergence (which is just the derivative 
thanks to the assumption) of $A \, \frac{1}{r}$:
\[
1 = \int_S - \frac{A}{r^2} d\mathbf{s} = - \frac{4 \pi \slashed{r^2} 
A}{\slashed{r^2}}
\]
From which we conclude:
\[
A = - \frac{1}{4 \pi}
\]
\end{document}