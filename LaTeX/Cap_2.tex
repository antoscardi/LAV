\chapter{Electromagnetism}

The ARTVA instrument in transmitter and receiver 
mode is a magnetic dipole. In order to formulate 
a coherent mathematical model, it is necessary to 
report some results of electromagnetic theory. 
Firstly, we identify and name the fundamental 
electromagnetic physical quantities.

\begin{table}[h!]
    \centering
    \begin{tabular}{lll}
        \hline
        Symbol & Description & Units \\
        \hline
        $\mathbf{E}$ & Electric field intensity & V/m \\
        $\mathbf{D}$ & Electric displacement field & C/m$^2$ \\
        $\mathbf{H}$ & Magnetic field intensity & A/m \\
        $\mathbf{B}$ & Magnetic flux density & T \\
        $\mathbf{J}$ & Current density & A/m$^2$ \\
        $\mathbf{A}$ & Magnetic vector potential & V s/m \\
        $\rho$ & Volume charge density & C/m$^3$ \\
        $\epsilon$ & Permittivity of the medium & F/m \\
        $\mu$ & Permeability of the medium & H/m \\
        $c$ & Speed of light in vacuum & m/s \\
        \hline
    \end{tabular}
    \caption{List of electromagnetic physical 
    quantities and their descriptions.}
    \label{tab:symbols}
\end{table}

\section{Maxwell's Equations}

We postulate Maxwell's equations in a simple 
(linear, isotropic, and homogeneous) medium in 
phasor notation \ref{Phasor Notation}, which have 
been discovered experimentally \cite{book-magnetism}:

\begin{align}
    \nabla \times \mathbf{E} &= -j\omega \mathbf{B} 
    \label{eq:curl_E} \\
    \nabla \times \mathbf{H} &= \mathbf{J} + j\omega 
    \epsilon \mathbf{E} \label{eq:curl_H} \\
    \nabla \cdot \mathbf{E} &= \frac{\rho}{\epsilon} 
    \label{eq:E} \\
    \nabla \cdot \mathbf{B} &= 0 \label{eq:B}
\end{align}

In these equations, the space-coordinate arguments 
have been omitted for simplicity. The fact that 
the same notations are used for the phasors as 
are used for their corresponding time-dependent 
quantities should create little confusion because 
we will deal exclusively with sinusoidal vector 
fields.

From Maxwell's equation \ref{eq:B}, we know that 
the magnetic flux density \(\mathbf{B}\) is solenoidal 
(zero divergence). Then, \(\mathbf{B}\) can be expressed 
as the curl of another vector field using the 
Null Theorem \ref{eq:null}, obtaining \ref{eq:B}:

\begin{equation}
    \mathbf{B} = \nabla \times \mathbf{A}
    \label{curlA}
\end{equation}

Also, \(\mathbf{B}\) relates to the magnetic field 
intensity \(\mathbf{H}\) through the permeability of 
the medium \(\mu\):

\begin{equation}
    \mathbf{B} = \mu \mathbf{H}
    \label{eq:BH}
\end{equation}

Another useful form of Maxwell's first equation 
\ref{eq:curl_E} can be found by substituting 
\ref{curlA}:

\begin{align}
    \nabla \times \mathbf{E} = - j \omega 
    (\nabla \times \mathbf{A}) &= -\nabla \times 
    j \omega \mathbf{A} \nonumber \\
    \nabla \times \left(\mathbf{E} + 
    j \omega \mathbf{A}\right) &= 0 \nonumber
\end{align}

Since the sum of vector fields is itself a vector 
field, \(\mathbf{E} + j \omega \mathbf{A}\) is a 
vector field, and we can define a \textbf{scalar} 
field, the electric potential \(V\), such that:

\begin{equation}
    \mathbf{E} + j \omega \mathbf{A} = -\nabla V
    \label{eq:V}
\end{equation}

If the curl of a vector field is zero, a scalar 
field exists whose gradient gives the vector field.

\subsection{Conservation of Charge Principle}

The principle of conservation of charge states 
that the net charge within a closed system remains 
constant over time, meaning no charge can be 
created nor destroyed \cite{book-magnetism}:

\begin{equation}
    \nabla \cdot \mathbf{J} = -\frac{\partial \rho}
    {\partial t}
\end{equation}

The current density \(\mathbf{J}\) is defined as:

\begin{equation}
    \mathbf{J} = N q \mathbf{u}
\end{equation}

where \(N\) is the number of charge carriers per 
unit volume, \(q\) is the charge of each carrier, 
and \(\mathbf{u}\) is the drift velocity of the 
charge carriers.

This means that if a current flows out of a volume, 
the charge density inside the volume must decrease 
at a rate equal to the current. The current leaving 
the volume is the flux of the current density 
through surface \(S\):

\begin{equation}
    I = \oint_S \mathbf{J} \cdot d\mathbf{s}
    \label{eq:defI}
\end{equation}

\section{Wave Equation for Magnetic Vector Potential}

In order to determine the intensity of the magnetic 
field, we first need to find an expression for the 
magnetic vector potential \(\mathbf{A}\), called the 
wave equation. Starting from Maxwell's equations, 
we find the wave equation by substituting \ref{curlA} 
and \ref{eq:BH} into the second Maxwell equation 
\ref{eq:curl_H}:

\[
    \nabla \times \nabla \times \mathbf{A} = 
    \mu \mathbf{J} + j\omega\epsilon \mu \mathbf{E}
\]

Then we substitute \ref{eq:V} for $\mathbf{E}$ and 
use the Laplacian \ref{eq:laplacian} on the left side:

\[
    \nabla (\nabla \cdot \mathbf{A}) - \nabla^2 
    \mathbf{A} = \mu \mathbf{J} + j\omega \epsilon 
    \mu \left(-\nabla V - j \omega \mathbf{A}\right)
\]

The definition of a vector requires the specification 
of both its curl and its divergence. Although the curl 
of $\mathbf{A}$ is designated $\mathbf{B}$ in 
\ref{curlA}, we are still at liberty to choose its 
divergence to simplify the expression 
\cite{book-magnetism}:

\[
    \nabla \cdot \mathbf{A} = - j \omega \epsilon 
    \mu V
\]

Finally, rearranging the terms and substituting 
the square of $j$, we get:

\[
    \nabla^2 \mathbf{A} + \omega^2 \epsilon \mu 
    \mathbf{A} = -\mu \mathbf{J}
\]

This is the wave equation for the magnetic vector 
potential \(\mathbf{A}\):

\begin{equation}
    \nabla^2 \mathbf{A} - k^2 \mathbf{A} = 
    -\mu \mathbf{J}
    \label{eq:wave}
\end{equation}

where \(k = \omega \sqrt{\mu \epsilon}\) is the wave 
number, which characterizes the propagation of the 
electromagnetic wave in the medium.
\subsection{Finding the Potential by Solving the Wave Equation}

Since both \(\mathbf{A}\) and \(\mathbf{J}\) are vector 
fields, the wave equation \ref{eq:wave} can be written 
for each component of \(\mathbf{A}\):

\begin{equation}
    \nabla^2 
    \begin{pmatrix} 
        A_x \\ A_y \\ A_z 
    \end{pmatrix} = 
    k^2 
    \begin{pmatrix} 
        A_x \\ A_y \\ A_z 
    \end{pmatrix} - 
    \mu 
    \begin{pmatrix} 
        J_x \\ J_y \\ J_z 
    \end{pmatrix}
    \label{eq:wavematrix}
\end{equation}

In order to solve \ref{eq:wavematrix}, we can use the 
Green function for Poisson's equation \ref{eq:poisson} 
for each of the components when $k = 0$ (the case of 
static fields). Then, the solution is given by the 
formula:

\begin{equation}
    \mathbf{A} = \frac{\mu}{4 \pi} \int_V 
    \mathbf{J} \frac{e^{-jkr}}{r} dv
    \label{eq:solA}
\end{equation}

by using the superposition principle and by finding the 
solution of the time-dependent differential equation.

\section{Magnetic Dipole}

\begin{figure}
    \centering
    \begin{tikzpicture}
        % Draw the 3D coordinate system
        \draw[axis] (0,0,0) -- (3,0,0) 
        node[anchor=north east]{$z$};
        \draw[axis] (0,0,0) -- (0,3,0) 
        node[anchor=north west]{$x$};
        \draw[axis] (0,0,0) -- (0,0,4) 
        node[anchor=south]{$y$};
        % Coordinates
        \coordinate (O) at (0,0,0);
        \coordinate (R) at (3,2,1);
        % Line from p to b
        \draw[thick] (1,0,0) -- (R) 
        node[midway, below, yshift=-0.1cm]{$r_1$};
        % Draw the point r
        \fill (R) circle (2pt);
        \node[above right] at (R) {$p$};
        % Draw and label the angle theta
        \draw (0,1) arc[start angle=90,end angle=58,,radius=2cm];
        \node at (0.5,0.7) {$\theta$};
        % Label in space called b
        \node at (1,0,1.5) {$I$};
        % Draw a circle in the xz-plane
        \begin{scope}[canvas is xz plane at y=0]
            \draw[thick, color=blue!80!black] 
            (0,0) circle [radius=1];
            \draw[-stealth, color=blue!80!black] 
            (0,1) arc[start angle=90,end angle=45,radius=1];
        \end{scope}
        % Draw the line from origin to point r
        \draw[thick] (O) -- (R) node[midway, above]{$r$};
    \end{tikzpicture}
    \caption{Magnetic dipole representation}
    \label{fig:dipole}
\end{figure}

We have a small filament loop of radius \( b \), 
carrying an AC current \( I(t) = I \cos(\omega t) \) 
as shown in Figure \ref{fig:dipole}. If \( S \) is the 
cross-section area of the wire and \( dl \) a 
differential length, we have \( \mathbf{J} \perp 
\mathbf{s} \), the normal to the area and the volume:

\[
    dv = S \, dl
\]

In order to determine the magnetic field intensity 
\(\mathbf{H}\) at a certain point in space \( p \), 
we need to compute the magnetic vector potential 
\(\mathbf{A}\) first \cite{book-magnetism}. Since the 
charges move only in the thin wire, they are located 
only in the wire region, then from the definition of 
current \ref{eq:defI}:

\[
    I = S \, \mathbf{J}
\]

The volume integral in \ref{eq:solA} becomes:

\[
    \int_V \mathbf{J} \, dv =  \int_V \mathbf{J} \, S \, dl
\]

Then, we substitute \ref{eq:defI} and change the 
integral type since it is now the sum over the length, 
so the formula becomes (the current must flow in a 
closed path):

\begin{equation}
    \mathbf{A} = \frac{\mu_0 I}{4\pi} \oint_C 
    \frac{d\mathbf{l}}{r_1} e^{-j k r_1}
    \label{sol_A_dipole}
\end{equation}

where $r_1$ is the distance between the point $p$ and 
the charges (source element) and $d\mathbf{l}$ is a 
vector tangent to the loop of differential length $dl$.

\subsubsection{Assumption} \label{Assumption}

We can then simplify \ref{sol_A_dipole} by considering 
the radius $b$ to be small enough, such that 
$r_1 - r \approx 0$. Then by adding and subtracting $r$ 
from the power of the exponential:

\[
    e^{-j k r_1} = e^{-j k (r_1 + r - r)} = 
    e^{-j k r} \, e^{-j k (r_1 - r)}
\]

Then by using Taylor approximation on the second 
exponential ($ x = r_1 - r \approx 0$) we obtain:

\[
    e^{-j k r_1} = e^{-j k r}\, \left[ 1 - j k (r_1 - r) \right]
\]

Then, we substitute this result in \ref{sol_A_dipole} 
and simplify:

\[
\begin{aligned}
    \mathbf{A} &= \frac{\mu_0 I}{4 \pi} e^{-j k r} 
    \left[ 1 - j k (r_1 - r) \right] \oint_C 
    \frac{d\mathbf{l}}{r_1} \\
    &= \frac{\mu_0 I}{4 \pi} e^{-j k r} \left( 
    \oint \frac{d\mathbf{l}}{r_1} - j k \oint 
    (r_1 - r) \frac{d\mathbf{l}}{R_1} \right)
\end{aligned}
\]

Since the integral of \(d\mathbf{l}\) over a closed loop 
is zero, because we have considered a small loop $b \to 0$:

\[
    \oint_C d\ell = 2\pi b \quad \rightarrow \quad 
    \oint_C d\ell \rightarrow 0
\]

Then we obtain:

\begin{equation}
    \mathbf{A} = \frac{\mu_0 I}{4 \pi} e^{-j k r} 
    \left[ \left(1 + j k r \right) \oint 
    \frac{d\mathbf{l}}{r_1}\right]
    \label{eq:sol_A_simp}
\end{equation}
\subsubsection{Assumption Quasi-Static Field/Near Field Zone}
If we consider a region near the magnetic dipole, we 
obtain quasi-static fields. We defined the wave number 
$k$ as:

\begin{equation}
    k = \omega \sqrt{\mu \epsilon}
    \label{eq:k}
\end{equation}

Electromagnetic waves propagate with velocity $u$ 
(speed of light in vacuum) \cite{book-magnetism}:

\begin{equation}
    u = \frac{1}{\sqrt{\mu \epsilon}}
    \label{eq:u}
\end{equation}

Then, by inverting \ref{eq:u} and substituting in 
\ref{eq:k}, we can write $k$ as:

\begin{equation}
    k = \frac{\omega}{u}
    \label{eq:k2}
\end{equation}

From wave theory $f = \frac{\omega}{2\pi}$ and 
$\lambda = \frac{u}{f}$, we obtain another expression 
for $u$:

\begin{equation}
   u = \frac{\lambda \, \omega}{2 \, \pi}
\end{equation}
\label{eq:u2}

\noindent Therefore, we can substitute \ref{eq:u2} in 
\ref{eq:k2}:

\begin{equation}
    k = \frac{2 \pi}{\lambda}
    \label{eq:k3}
\end{equation}

To simplify the expression for $\mathbf{A}$ 
\ref{sol_A_dipole}, we make the assumption that 
$k \, r \ll 1$, and if we substitute the found 
expression of $k$ \ref{eq:k3}:

\[
    k \, r \ll 1 \implies \frac{2\pi r}{\lambda} 
    \ll 1 \implies r \ll \frac{\lambda}{2\pi}
\]

This means that $r$ needs to be small in comparison 
to $\lambda$. If this is the case:

\[
    e^{-j k r} \approx e^0 = 1
\]

We eliminate completely the time dependence and 
obtain the expression for $\mathbf{A}$:

\begin{equation}
    \mathbf{A} = \frac{\mu_0 I}{4\pi} \oint_C 
    \frac{d\mathbf{l}}{r_1}
    \label{A_approx}
\end{equation}

In the ARTVA case, the standard operating frequency 
is \( f = 475 \, \text{kHz} \) and the optimal range 
of the instrument is \( <80 \, \text{m} \). Then in 
the worst case, when \( r = 80 \, \text{m} \), we 
obtain the approximation \( k \, r = 0.79 \).

\subsubsection{Symmetry}

In the particular case of a magnetic dipole, the 
magnetic vector potential $\mathbf{A}$ is symmetric 
with respect to the \(x\)-axis, therefore independent 
to the $\phi$ angle \ref{Spherical Coordinates}. 
This is true since we can choose freely the $z$-axis 
and $y$-axis orientation in space around the loop. 
Then we can choose the point $\mathbf{p}$ to lie on 
the $zx$-plane or the $yx$-plane; in both cases, we 
will obtain that one of the two $d\mathbf{l}$ 
components $d\mathbf{l_z}$ and $d\mathbf{l_y}$ will 
cancel themselves out as we integrate over the loop.

For example, if we consider the point to lie on the 
$yx$-plane, then take a point on the loop where 
$d\mathbf{l}$ is and its symmetric w.r.t. the 
$y$-axis, the component $d\mathbf{l_y}$ of the first 
will cancel itself out with the one of the second.

We can write the length of a circumference as 
$l = r \, \alpha$, where $\alpha$ is the subtended 
angle by the length $l$ and $r$ the radius. In 
addition, we express $\mathbf{e_\phi}$ using the 
Cartesian basis:

$$
    \mathbf{e_\phi} = - \sin\phi \, \mathbf{e_y} 
    + \cos\phi \, \mathbf{e_z}
$$

Then, $d\mathbf{l}$ magnitude depends on the 
differential angle $d\phi$ and the radius $b$, and 
has the same direction as $\mathbf{e_\phi}$:

\begin{equation}
    d\mathbf{l} = b \, d\phi \, \mathbf{e_\phi} 
    = b \, d\phi \left(- \sin\phi \, \mathbf{e_y} 
    + \cos\phi \, \mathbf{e_z}\right)
    \label{eq:dl}
\end{equation}

For every $d\mathbf{l}$, there is another 
symmetrically located differential length element on 
the other side of the $y$-axis that will contribute 
an equal amount to $\mathbf{A}$ in the $\mathbf{e_z}$ 
direction but will cancel the contribution of 
$d\mathbf{l}$ in the $\mathbf{e_y}$ direction. 
Since $\mathbf{e_z} = \mathbf{e_\phi}$, if point $P$ 
lies on the $yx$-plane, equation \ref{A_approx} can 
be written as:

\begin{equation}
    \mathbf{A} = \mathbf{e_\phi} \frac{\mu I b}{4 \pi} 
    \int_{0}^{2\pi} \frac{\cos \phi}{r_1} \, d\phi
    \label{eq:A_final}
\end{equation}

\subsubsection{Computing the Integral in Spherical Coordinates}

Firstly, we find $r_1$ by applying the law of cosines 
on the triangle $OPP'$, Figure \ref{fig:dipole2}:

\begin{figure}
    \centering
    \begin{tikzpicture}
        % Draw the 3D coordinate system
        \draw[axis] (0,0,0) -- (4,0,0) 
        node[anchor=north east]{$z$};
        \draw[axis] (0,0,0) -- (0,4,0) 
        node[anchor=north west]{$x$};
        \draw[axis] (0,0,0) -- (0,0,5) 
        node[anchor=south]{$y$};
        % Coordinates
        \coordinate (O) at (0,0,0);
        \coordinate (R) at (0,4,4);
        \coordinate (P') at (1.7,0,1.2);
        % Line from p to b
        \draw[thick] (P') -- (R) node[right, yshift=0.1cm]{};
        % Draw the point 
        \fill (R) circle (2pt);
        \fill (P') circle (2pt);
        \node[above right] at (R) {$P$};
        % Draw the angle theta between r and y-axis
        \draw[->] (0,1.5,0) arc[start angle=90, 
        end angle=115, radius=2cm];
        \node at (-0.3,1.7,0) {$\theta$};
        \draw[bend right,->] (0,0,1) to 
        node [auto] {} (0.7,0,0.5);
        \node at (0.7,0.6,1) {$\beta$};
        \draw[bend right,->] (0.4,0,0.3) to 
        node [auto] {} (0,0.5,0.5);
        % Label in space called b
        \node at (1.8,0,1.8) {$b$};
        \node at (2.3,0,2) {$P'$};
        \node at (0.7,0,1.2) {$\phi$};
        % Draw a circle in the xz-plane
        \begin{scope}[canvas is xz plane at y=0]
            \draw[thick, color=blue!80!black] 
            (0,0) circle [radius=2];
        \end{scope}
        % Draw the line from origin to point r
        \draw[thick] (O) -- (R) 
        node[midway, above right, yshift=0.4cm]{};
        \draw[thick] (O) -- (P') 
        node[midway, above right]{};
    \end{tikzpicture}
    \caption{Magnetic dipole representation with the $OPP'$ triangle}
    \label{fig:dipole2}
\end{figure}

We start with the equation for \( r_1 \):
$$
    r_1^2 = r^2 + b^2 - 2 r b \cos\beta
$$
Since we are on the \( xy \)-plane, we can write the 
\( \mathbf{b} \) and \( \mathbf{r} \) vectors as:
$$
    \mathbf{r} = r \sin \theta \, \mathbf{e_y} 
    + r \cos \theta \, \mathbf{e_x}
$$
$$
    \mathbf{b} = b \cos \phi \, \mathbf{e_y} 
    + b \sin \phi \, \mathbf{e_z}
$$
Then we find the angle \(\beta\) between 
\( \mathbf{r} \) and \( \mathbf{b} \):
$$
    \cos \beta = \frac{\mathbf{b} \cdot \mathbf{r}}{r b} 
    = \sin \theta \cos \phi
$$
We have obtained a formula for \( r_1 \):
$$
    r_1 = \sqrt{r^2 + b^2 - 2 r b \sin \theta \cos \phi}
$$
Simplifying further, we get:
$$
    r_1^2 = r^2 \left( 1 + \frac{b^2}{r^2} - 2\, 
    \frac{b}{r} \sin \theta \cos \phi \right)
$$
Using the same assumption as before, that the loop is 
very small with respect to \( r \) (i.e., \( b \ll r \) 
and therefore \( b^2 \ll r^2 \)), we can write:
$$
    r_1 \approx r \left( 1 - 2 \, \frac{b}{r} \, 
    \sin \theta \cos \phi \right)^{1/2}
$$
Then we compute the inverse \( \frac{1}{r_1} \) and 
use the Taylor approximation to the first derivative, 
considering \( x = 2 \, \frac{b }{r} \, \sin \theta 
\cos \phi \), which tends to zero, we obtain:

\begin{equation}
    \frac{1}{r_1} \approx \frac{1}{r} 
    \left( 1 + \frac{b}{r} \sin \theta \cos \phi \right)
    \label{eq:r_1}
\end{equation}

Now, substituting \ref{eq:r_1} in \ref{eq:A_final}, we 
can calculate the integral for \( \mathbf{A} \) over 
the entire loop:

$$
    \mathbf{A} = \mathbf{e_\phi} \frac{\mu I b}{4 \pi} 
    \int_{0}^{2\pi} \left( 1 + \frac{b}{r} \sin \theta 
    \cos \phi \right) \cos \phi \, d\phi
$$

Since \( b \), \( r \), and $\theta$ do not depend on 
$\phi$, the first integral is zero, and the second one 
gives $\pi$:

\begin{enumerate}[label=(\arabic*)]
    \item 
    \parbox{\textwidth}{
    \[
        \int_0^{2\pi} \cos \phi \, d\phi = \sin \phi \Big|_0^{2\pi} = 0
    \]
    }

    \item 
    \parbox{\textwidth}{
    \[
        \int_{0}^{2\pi} \cos^2 \phi \, d\phi = \int_{0}^{2\pi} 
        \frac{1 + \cos(2\phi)}{2} \, d\phi = \frac{1}{2} \cdot 2\pi 
        + \frac{1}{2} \cdot 0 = \pi
    \]
    }
\end{enumerate}

Therefore, we obtain:

\begin{equation}
    \mathbf{A} = \mathbf{e_\phi} \frac{\mu I b^2}{4 \, r^2} 
    \sin \theta
    \label{eq:A_spher}
\end{equation}

\subsubsection{Magnetic Field Intensity $\mathbf{H}$}
Finally, we can now obtain an expression in spherical 
coordinates of the magnetic field intensity 
$\mathbf{H}$, by first finding $\mathbf{B}$ using 
\ref{curlA} and then inverting \ref{eq:BH}.
\\
We compute the curl of $\mathbf{A}$ in spherical 
coordinates \ref{curl_spher}:

\[
    \mathbf{B} = \nabla \times \mathbf{A} = \frac{1}{r^2 \sin \theta} 
    \begin{vmatrix}
        \mathbf{e_r} & \mathbf{e_\theta} r & \mathbf{e_\phi} r \sin \theta \\
        \frac{\partial}{\partial r} & \frac{\partial}{\partial \theta} & 
        \frac{\partial}{\partial \phi} \\
        0 & 0 & r \sin \theta A_\phi
    \end{vmatrix} =
\]
\[
    = \frac{1}{r^2 \sin \theta} \left( \mathbf{e_r} 
    \frac{\partial}{\partial \theta} (r \sin \theta A_\phi) 
    - \mathbf{e_\theta} \frac{\partial}{\partial r} 
    (r \sin \theta A_\phi) \right)
\]

where $A_\phi$ is the magnitude of the vector field 
found in \ref{eq:A_spher}, while $A_r$ and $A_\theta$ 
are zero since the potential has only the 
$\mathbf{e_\phi}$ direction. Substituting 
\ref{eq:A_spher}:

\[
\begin{aligned}
    \mathbf{B}&= \frac{1}{r^2 \sin \theta} 
    \frac{\mu b^2 I}{4 \pi} \left(\mathbf{e_r} 
    \frac{2}{r} \cos \theta \sin \theta + 
    \mathbf{e_\theta} r \sin^2 \theta \, r^{-2}\right) =
    \\
     &= \frac{\mu I b^2 }{4 \, r^3} \left( 
     \mathbf{e_r} 2 \cos \theta + \mathbf{e_\theta} 
     \sin \theta \right)
\end{aligned}
\]

Then, by inverting equation \ref{eq:BH}, we obtain the 
final expression for $\mathbf{H}$:

\begin{equation}
    \mathbf{H} = \frac{ I b^2}{4 \, r^3} \left( 
    \mathbf{e_r} 2 \cos \theta + \mathbf{e_\theta} 
    \sin \theta \right)
    \label{eq:H_spheric}
\end{equation}

which can also be expressed using the Cartesian unit 
vectors \((\mathbf{e}_x, \mathbf{e}_y, \mathbf{e}_z)\) 
by substituting \(\mathbf{e}_r\) with \ref{e_r} and 
\(\mathbf{e}_\theta\) with \ref{e_theta}:

\[
    \mathbf{H} = \frac{I b^2}{4 \pi \, r^3} \left[ 
    (2\cos^2\theta - \sin^2\theta) \mathbf{e}_x 
    + 3\cos\theta \sin\theta \cos\phi \, \mathbf{e}_y 
    + 3\cos\theta \sin\theta \sin\phi \, \mathbf{e}_z \right]
\]

or in vector form:

\[
    \mathbf{H} = \frac{I b^2}{4 \, r^3} 
    \begin{bmatrix}
        2\cos^2\theta - \sin^2\theta \\
        3\cos\theta \sin\theta \cos\phi \\
        3\cos\theta \sin\theta \sin\phi
    \end{bmatrix}
\]

We can finally find the expression for $\mathbf{H}$ 
using only Cartesian coordinates by inverting the 
equations in \ref{Conversion from Spherical to 
Cartesian Coordinates}:

\[
    \mathbf{H} = \frac{I b^2}{4 \, r^5} 
    \begin{bmatrix}
        2x^2 - y^2 - z^2 \\
        3xy \\
        3xz
    \end{bmatrix}
\]

which has been derived by exploiting the Pythagorean 
identity.
