\documentclass[main]{subfiles}
\begin{document}
\section{Conclusions}
%RILEGGERE LE CONCLUSIONI
In this work, we modify the PSO algorithm to address the challenging
task of multi-source localization using a decentralized swarm of UAVs.
The results show that, in general, the algorithm produces reliable
performance with precise estimates and fast convergence times, even in
complex configurations. However, as an inherently heuristic approach,
the PSO is not guaranteed to always converge, and certain situations
highlight its potential drawbacks.

\noindent\\
One of the main challenges we face is overcoming the superposition in
signal strength from multiple sources or victims. This is our primary
focus, as it inherently complicates the identification of individual
sources. To address this issue, we implement a strategy (the PSO) that
allows the swarm to localize multiple sources based on the aggregated
signal, effectively bypassing the need to separate and identify
individual ones. While this approach proves successful in enabling the
swarm to locate sources in challenging scenarios, it introduces a
trade-off: the certainty of finding a source is reduced because the PSO
heuristic approach relies on some degree of randomness and the exclusion
zone mechanism.

\noindent\\
Another notable drawback is that the algorithm occasionally fails in
scenarios where UAVs are repeatedly attracted to the same sources,
preventing effective exploration of the search space. This limitation
prevents the swarm's ability to identify and reach faraway
sources, when near others. This behavior is an intrinsic limitation of 
the problem. Additionally, when sources
are positioned too closely together, as demonstrated in Case 3, the
algorithm struggles to differentiate between them, resulting in reduced
accuracy unless additional UAVs are introduced to improve coverage.

\noindent\\
One potential enhancement in our algorithm is the employment of adaptive mechanisms 
for tuning the PSO parameters, such as inertia weight 
which could dynamically adjust based on the swarm's progress or environmental 
conditions, improving convergence speed and accuracy.
In addition, investigating hybrid optimization methods that combine 
the PSO with local search or other heuristic approaches could provide a more 
comprehensive solution for complex multi-source localization tasks.

\noindent\\
On the other hand, the algorithm's strengths lie in its adaptability and
the benefits of a decentralized approach. The exclusion zone mechanism
enables the swarm to dynamically adjust its search strategy based on the
locations of previously detected sources, promoting faster convergence
and improved accuracy. This adaptability is especially valuable in
multi-source localization problems, where the location of all different
sources rapidly is crucial.
Furthermore, our work also introduces another novel aspect: 
the implementation of the PSO algorithm to provide control command velocities 
to the UAVs in real time, integrating the PSO into the control
loop.

\noindent\\
The decentralized nature of our approach offers further advantages. By
removing the reliance on a central computation unit, the system becomes
more robust to communication noise and failures. This autonomy allows
the UAVs to operate effectively under challenging conditions, such as
mountainous terrain or adverse weather, where communication may be
intermittent or delayed. While decentralization may lead to slightly
slower convergence and higher estimation errors due to partial
information exchange, it remains an effective solution in scenarios
where centralized coordination is impractical or resource-intensive.

\noindent\\
In conclusion, our implementation demonstrates that while the modified PSO
algorithm encounters limitations in specific cases, it provides a
powerful and flexible tool for decentralized multi-source localization.
By addressing the challenge of signal overlap, our algorithm proves to be
both effective and efficient in complex conditions and environment-associated
with avalanche victims' localization.
\end{document}