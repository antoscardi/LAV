\documentclass[main]{subfiles}
\begin{document}
\section{Introduction}
\label{intro}

\subsection{Previous Works}
In the context of S\&R\footnote{S\&R stands for Search and Rescue.} operations, 
the support of robots has become more and more extensive in recent years 
\cite{sr0, sr1} and in particular the necessity and advantages of using 
\textit{decentralized multi-agent} systems \cite{multiagent}. A particular field 
of S\&R missions focuses on high mountain scenarios, in which UAVs 
\footnote{UAV stands for Unmanned Aerial Vehicles.} are tasked with localizing 
avalanche victims. These missions involve the smart collaboration between UAVs 
and humans also developed in the context of European-founded projects such as 
the SHERPA one \cite{sherpa}.

\noindent\\Different types of sensors can be used to achieve the localization goal, but the 
fastest and most accurate ones are electromagnetic sensors, which can operate 
also in noisy environments \cite{sensors}. Among these, the ARTVAs\footnote{ARTVA 
stands for the Italian “Apparecchio di Ricerca dei Travolti in VAlanga”.} are the 
most commonly used in both human-conducted and robot-conducted operations.

\noindent\\
Many studies have analyzed and formulated strategies and algorithms in the 
particular context of S\&R operations using UAVs for avalanche victims 
\cite{main, precedente, pre-main}. Furthermore, in order to address 
the localization problem, the authors of \cite{similar-main} and 
\cite{first-model} propose a robocentric SLAM approach in the broader robotics 
context. Note that in \cite{similar-main} they also take into consideration the 
\textit{multiple victims} case.

\noindent\\Avalanche victims rescue missions are challenging with respect to 
other S\&R operations because of some challenging aspects. In particular, the 
time constraint is quite demanding since the survival chances of avalanche 
victims diminish rapidly with time. Victims buried under avalanches have a 93\% 
survival rate within the first 15 minutes, which drops to 25\% after 45 minutes 
due to hypothermia \cite{survival} and plateaus at 21\% from 60 min to 180 
minutes \cite{survival2}.

\noindent\\The common human-conducted avalanche victims S\&R technique using ARTVA 
technology relies on three phases. In the first phase, rescuers search for the 
first valid electromagnetic signal, which can be detected at distances ranging 
from 20 meters for single-antenna receivers to 80 meters for triple-antenna ones. 
In the second phase, rescuers are trained to interpret the magnetic field data 
and follow standard procedures in order to follow the magnetic field direction and 
find the victim's position. In the final phase, rescuers dig to save the buried 
individual.

\noindent\\Despite the effectiveness of this technique, it requires a significant amount of 
time due to the challenges of traversing avalanche terrain. Additionally, 
rescuers walking on unstable snow face the tangible risk of triggering a 
secondary avalanche \cite{first-model}. For these reasons, and given the 
additional advantage of typically not encountering obstacles in high mountain 
scenarios, the use of intelligent \textit{autonomous} UAVs results in a faster 
and safer search when compared to human rescuers, as shown in \cite{sr, sr2, sr3}.

\noindent\\Therefore, this work aims to build upon previous efforts to 
\textit{automate} and improve the efficiency of the second phase by developing a 
mathematical/algorithmic framework for solving the localization problem of not 
just one, but \textit{multiple avalanche victims}, using \textit{multiple 
decentralized} agents (UAVs). This approach aims to reduce computational 
complexity without sacrificing accuracy and convergence time.

\subsubsection{Magnetic Dipole Localization}
In the context of a single magnetic source, numerous studies have focused 
on the localization task; in \cite{single_closed_formula_position} 
a straightforward closed-form solution is found by leveraging the analytical 
expressions of the magnetic field vector and the magnetic gradient tensor, 
regardless of the singularity of the magnetic gradient tensor matrix.

\noindent\\In the field of TMI\footnote{TMI stands for Total Magnetic Intensity} 
gradient surveys, the acquisition of magnetic gradient tensor data 
has seen significant advancements.
In \cite{NSS_analysis2}, new methods have been 
developed for inverting gradient tensor surveys, 
for several elementary, but useful, models. 
These include point pole, line 
of poles, point dipole (sphere), line of dipoles (horizontal cylinder), thin and thick 
dipping sheets, sloping step, and contact models.
A key insight is the use of eigenvalues and associated eigenvectors of the magnetic
tensor to obtain the NSS \footnote{NSS stands for Normalized Source Strength.},
which is a particularly useful rotational invariant that peaks 
directly over 2D sources, and it is independent of magnetization 
direction.

\noindent\\The NSS has been employed not only in geological applications but also in the  
the detection, location, and classification of magnetic objects, such as naval 
mines, UXO, shipwrecks, archaeological artifacts, and buried drums \cite{NSS_analysis2}.

\noindent\\In \cite{NSS_formula_important}, the authors use the spatial relation between the 
magnetic target and observation points derived from the NSS for accurately locating 
a magnetic target, achieving nearly error-free results in the absence of noise.
In addition, methods that use the NSS for magnetic sources localization, for different models
and dimensions is done in \cite{NSS_single_different_dimensions}. 

\noindent\\
In \cite{NSS_single_localization}, a key aspect of the NSS is identified by the fact
that it is only dependent on the distance between the source and target position;
since an analytical expression of the magnetic gradient tensor and a closed-form 
localization formula are derived.

\noindent\\
In \cite{NSS_analysis} as well, it is stressed that the NSS is proportional 
to a constant normalized by the distance between observation and source. 
It is independent of magnetization direction and satisfies Euler’s homogeneity equation, 
this allows the authors to use Euler deconvolution of the NSS to estimate source location.  

\noindent\\
Instead, the more specific multi-magnetic source localization task 
has been a difficult problem and different strategies have been conducted.
\noindent
In recent years, many studies have performed an in-depth analysis of rotational 
invariants of the magnetic gradient tensor, such as \cite{multiple_plots} and 
\cite{multiple_real_plots_invariants}.
In the latter, a new method, named NED\footnote{NED stands for New Edge Detection} 
has been proposed, which is free from geomagnetic interference and provides 
abundant information, and it is compared with other methods such as the tilt angle 
(ratio of vertical to horizontal magnetic field components),
theta map, etc.

\noindent\\
In this context of detecting and localizing multiple dipole-like 
magnetic sources, the authors of \cite{multiple_DE_NSS_big_matrix} 
use the magnetic gradient tensor data:
the tilt angle is used to determine the number of sources, 
while the rotational-invariant NSS is used to estimate the horizontal coordinates. 
Then, they employ the DE algorithm to estimate the locations and moments of the sources.

\noindent\\
Lastly, the authors of \cite{multiple_spacecraft} use 
the principal invariants of the magnetic field gradient tensor
to determine the number and horizontal location of the magnetic sources, 
while Euler equations are used to compute the sources's depth.

\noindent\\
In our work, between all the different indicators in the literature,
we chose to employ the NSS in order to estimate the 
horizontal coordinates of the avalanche victims, 
since it is rotationally invariant and completely isotropic 
around the magnetic dipole.
We are not interested in estimating the burial depth of the victims,
even if \cite{not_import_formula_z} use a closed loop formula to find z,
it cannot be applied to the multiple sources case.


\subsubsection{Particle Swarm Optimization}
In this context of simultaneous localization of multiple magnetic 
dipole sources, like in this real-world application \cite{real_application}; 
different algorithms have been employed.
For instance, in \cite{DetectionMultiplMagnetiDipoleSources}, the authors
use a non-linear optimization method based on the Levenberg–Marquardt algorithm
to solve the problem, without prior knowledge of the number of 
dipoles in the 3-D detection region.

\noindent\\In this work, we propose the use of the PSO\footnote{PSO stands for Particle Swarm Optimization.} 
algorithm as a means to solve the multiple-sources localization task, 
which can be considered a continuous multi-modal optimization problem.

\noindent\\
The PSO\footnote{PSO stands for Particle Swarm Optimization.} algorithm 
was first introduced in 1995 \cite{PSO_original}, and it has since proven to be 
a simple but powerful tool for solving various optimization problems\cite{PSO_IMPORTANT},
among a lot of EAs \footnote{EA stands for Evolutionary Algorithm.}.
It has been used to solve various practical applications like simultaneous
localization of multiple odor sources \cite{12}, as well as detection of acoustic 
\cite{PSO_acustic}, thermal, or chemical/biological signals \cite{3}.
In the robotics field, it has not only been employed in source-seeking 
applications but also path-planning scenarios \cite{PSO_for_path_planning} and
deployment of communication networks \cite{PSO_for_communication}.

\noindent\\
In the standard PSO algorithm \cite{PSO_original}, a population of particles is 
randomly initialized to represent potential candidate solutions for the optimization problem. 
Each particle relies on two important pieces of information: its individual 
best solution, referred to as \textit{pbest}, and the global best across the entire population, 
referred to as \textit{gbest}. These two values guide the search direction of all particles 
over the search space. 
The evaluation of the \textit{pbest} for each particle and the \textit{gbest} for 
the entire population is determined by the function that needs to be optimized.

\noindent\\
However, when addressing continuous optimization problems with numerous local optima, 
classical optimization algorithms often require strict conditions 
and extensive computation times \cite{PSO_another}.
For this reason, the authors of  \cite{PSO_IMPORTANT} introduced a modified PSO algorithm 
where the original population is divided into multiple subpopulations, 
according to the order of particles.
The best particle within each subpopulation is recorded and then 
used in the velocity updating formula to replace 
the original global best particle of the entire population.
This strategy, based on the idea of multiple subpopulations, 
enables the algorithm to find several optima, including both
the global and local solutions, thanks to these distinct best particles
in each group.

\noindent\\
Both in the standard PSO \cite{PSO_original} and the multi-swarm modification \cite{PSO_IMPORTANT}
common limitations are the premature convergence of the algorithm 
and its tendency to become trapped in some local optima.
Instead in \cite{PSO_another}, the authors overcome these issue,
including the one of poor population diversity. 
They introduce a modification to \cite{PSO_IMPORTANT}, it is called 
ADPSO\footnote{ADPSO stands for Adaptive Dynamic Multi-Swarm Particle Swarm Optimization.}, 
and includes features for stagnation detection and spatial exclusion to mitigate these challenges. 
The dynamic division of the population into sub-swarms, 
coupled with a regrouping mechanism, avoids stagnation during the optimization process. 
Stagnant sub-swarms are rejuvenated with a newly generated particle, 
called the vitality particle, constructed from the historical information of the entire population. 
This approach maintains diversity and prevents sub-swarms from converging prematurely.

\noindent\\
The authors of \cite{PSO_another2} focus their attention to the 
exploration and exploitation aspects of the PSO algorithm, instead.
They enhance exploration and exploitation in a modified system,
named HCLPSO \footnote{HCLPSO stands for Heterogeneous Comprehensive 
Learning Particle Swarm Optimization.}, which divides the population 
into only two sub-swarms, one dedicated to exploitation and 
one dedicated to exploration. 
In the exploration-subpopulation, the solutions are generated by using the personal 
best experiences of the particles in the exploration subpopulation itself. 
In the exploitation subpopulation, the personal best experiences of the
entire swarm population are used to generate the solutions. 
In this way, since the exploration-subpopulation does not
learn from any particles in the other subpopulation, 
diversity is maintained in the exploration-subpopulation even when the 
exploitation-subpopulation converges prematurely.

\subsection{The ARTVA}
The ARTVA technology is composed of two different and easy switchable modalities: 
in \textit{receiver} mode, the instrument senses and processes the 
electromagnetic field emitted by the ARTVA \textit{transmitter} (carried by the 
avalanche victim).

\noindent\\
The magnetic field generated by the solenoid antenna of the instrument oscillates 
with a frequency of 457 kHz and its characteristics are defined in the standard 
ETS 300 718-1 \cite{eu_standard}, to ensure compatibility between different 
brands and models. To save batteries and facilitate detection, the magnetic field 
is transmitted in pulses of a tenth of a second every second \cite{first-model}.

\noindent\\
As will be discussed more in-depth in the following chapter, the magnetic field 
can be modeled as a three-dimensional vector field, which means that it assigns a 
certain intensity and direction to each point in space. Therefore, the main 
difference between different kinds of ARTVAs lies in \textit{"how much"} of this 
field they can measure. According to this criterion, the instruments can be 
divided into three different types \cite{first-model}:

\begin{itemize}
    \item ARTVAs \textit{with one reception antenna}: the oldest models, usually
 analog. The same antenna is used in both \textit{transmitter} and 
    \textit{receiver} mode. Therefore, only the projection of the magnetic field 
 on the antenna can be measured. This type of ARTVA is the most difficult to 
 use and the most time-consuming one.
    
    \item ARTVAs \textit{with two perpendicular reception antennas}: are based on 
 digital technology such as microprocessors. This type can measure only the 
 intensity and direction of the horizontal component of the field, only when 
 held in a horizontal position.
    
    \item ARTVAs \textit{with three mutually perpendicular reception antennas}: 
 also based on digital technology. Since these ARTVAs possess three 
 perpendicular antennas, they can measure the complete vector field. For this 
 reason, the instrument can be oriented w.r.t the magnetic field in any way.
\end{itemize}

\noindent\\
In this work, we will consider only new ARTVA transceivers (three antennas), 
which can achieve a search strip width in digital mode of 80 m and a maximum 
range in analog mode of 90 m \cite{manual}. Furthermore, it is important to point 
the transceiver in the direction of the avalanche, parallel to the slope. For 
this reason, in this work, two different types of trajectories have been 
considered.
\end{document}